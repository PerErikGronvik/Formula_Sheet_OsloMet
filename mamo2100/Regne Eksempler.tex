\documentclass{article}
\usepackage{amsmath}
\usepackage{siunitx}
\usepackage{multicol}
\usepackage{geometry}
\geometry{a4paper, left=0.5in, right=0.5in, top=0.5in, bottom=0.5in}

% Remove paragraph indentation
\setlength{\parindent}{0pt}

\begin{document}
\begin{center}
    \Large\textbf{Regne eksempler:} \\[0.8cm] %spacing
\end{center}
\begin{multicols}{2}
\section*{Relativity}


\begin{minipage}{\linewidth}
\textbf{Time dilation:} \\[-0.2cm]

A spaceship moves with velocity \( u = 0.8c \). The proper time measured on the spaceship is \( \Delta t_0 = 1 \, \si{\hour} \).

The time dilation formula is:
\[
\Delta t = \frac{\Delta t_0}{\sqrt{1 - \frac{u^2}{c^2}}}
\]

Substitute \( u = 0.8c \):
\[
\frac{u^2}{c^2} = 0.64 \quad \Rightarrow \quad \Delta t = \frac{1}{\sqrt{0.36}} \times 1 \, \si{\hour} = 1.6667 \, \si{\hour}
\]

The observer on Earth measures \( \Delta t \approx 1.67 \, \si{\hour} \).

\vspace{0.5cm}
\end{minipage}

\begin{minipage}{\linewidth}
\textbf{Length contraction:} \\[-0.2cm]

A spaceship at rest has a proper length of \( L_0 = 100 \, \si{\meter} \) and moves at \( u = 0.99c \).

The length contraction formula is:
\[
L = L_0 \sqrt{1 - \frac{u^2}{c^2}}
\]

Substitute \( u = 0.99c \):
\[
\frac{u^2}{c^2} = 0.9801 \quad \Rightarrow \quad L = 100 \sqrt{1 - 0.9801} = 100 \sqrt{0.0199}
\]

\[
L \approx 100 \times 0.141 = 14.1 \, \si{\meter}
\]

The observer on Earth measures the length of the spaceship as \( L \approx 14.1 \, \si{\meter} \).
\end{minipage}

\begin{minipage}{\linewidth}
\textbf{Relativistic energy and momentum:} \\[-0.2cm]

A particle with rest mass \( m = 1 \, \si{\kilogram} \) moves with velocity \( u = 0.6c \). Calculate its total energy \( E \) and momentum \( p \).

The Lorentz factor is:
\[
\gamma = \frac{1}{\sqrt{1 - \frac{u^2}{c^2}}}
\]

Substitute \( u = 0.6c \):
\[
\gamma = \frac{1}{\sqrt{1 - 0.36}} = \frac{1}{\sqrt{0.64}} = 1.25
\]

The total energy \( E \) is:
\[
E = \gamma mc^2
\]

\[
E = 1.25 \times 1 \, \si{\kilogram} \times (3.00 \times 10^8 \, \si{\meter\per\second})^2
\]

\[
E = 1.125 \times 10^{17} \, \si{\joule}
\]

The relativistic momentum \( p \) is:
\[
p = \gamma mu
\]

\[
p = 1.25 \times 1 \, \si{\kilogram} \times 0.6c
\]

\[
p = 1.25 \times 0.6 \times 3.00 \times 10^8 = 2.25 \times 10^8 \, \si{\kilogram\meter\per\second}
\]

Thus:
\[
E \approx 1.13 \times 10^{17} \, \si{\joule}, \quad p \approx 2.25 \times 10^8 \, \si{\kilogram\meter\per\second}
\]

\end{minipage}

\vspace{0.2cm}

\begin{minipage}{\linewidth}
\textbf{Exercise 1:} \\[-0.2cm]
Show that if $\Psi(x)$ is real-valued, then the expectation value of momentum $p$ is zero:
\[
\langle p \rangle = 0
\]
\end{minipage}

\vspace{0.2cm}

\begin{minipage}{\linewidth}
\textbf{Exercise 2:} \\[-0.2cm]
Show that the operator $\hat{T} = \frac{\hat{p}^2}{2m} = -\frac{\hbar^2}{2m} \frac{d^2}{dx^2}$ is Hermitian.
\end{minipage}

\textbf{Relativistic Work (with $v_x$):} \\[-0.2cm]
\[
W = \int_{v_1}^{v_2} \frac{m v_x \, dv_x}{\left( 1 - \frac{v_x^2}{c^2} \right)^{3/2}} = -\frac{mc^2}{2} \int_{\omega_1 = 1 - \frac{v_1^2}{c^2}}^{\omega_2 = 1 - \frac{v_2^2}{c^2}} \omega^{-3/2} \, d\omega = mc^2 \Big[ \frac{1}{\sqrt{\omega_2 = 1 - \frac{v_2^2}{c^2}}} - \frac{1}{\sqrt{\omega_1 = 1 - \frac{v_1^2}{c^2}}} \Big].
\]

\end{multicols}
\end{document}
