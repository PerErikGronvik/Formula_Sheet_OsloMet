\documentclass[a4paper,12pt]{article}
\usepackage{multicol}
\usepackage{geometry}
\usepackage{amsmath}
\usepackage{titlesec} % To customize section title size and spacing
\geometry{a4paper, left=0.5in, right=0.5in, top=0.5in, bottom=0.5in}
\setlength{\parindent}{0pt} % Remove paragraph indentation
\setlength{\columnsep}{1cm} % Set spacing between columns

% Adjust section title font size and spacing
\titleformat{\section}
  {\normalfont\normalsize\bfseries} % Set section title font to small and bold
  {}{0pt}{}
\titlespacing*{\section}{0pt}{0.3cm}{0.2cm} % Adjust space before and after section

\begin{document}

% Title section with small spacing
\begin{center}
    \large\textbf{Physics Concepts Dictionary} \\[0.2cm]
\end{center}

\vspace{-0.5cm} % Reduce space between title and content

\begin{footnotesize}

\begin{multicols}{3} % Set up 3 columns

\section*{Classical Mechanics}

\textbf{Galilean Transformation} \\
A principle in classical mechanics describing how measurements of position and velocity change between two observers moving at constant velocity relative to each other. It assumes that time is absolute and the same for all observers, which is not the case in relativity. \\

\textbf{Monotonic Function} \\
A function that is either entirely non-increasing or non-decreasing throughout its domain.  \\


\section*{Relativity}

\textbf{Time Dilation} \\
In special relativity, time dilation refers to the fact that time passes slower for an object in motion relative to a stationary observer. This effect becomes significant at speeds close to the speed of light. \\


\textbf{Lorentz Factor} \\
A factor that quantifies the amount of time dilation, length contraction, and the increase in relativistic mass for an object moving at a significant fraction of the speed of light. The faster the object moves, the larger the Lorentz factor. \\


\textbf{Relativistic Kinetic Energy} \\
At speeds close to the speed of light, the kinetic energy of an object grows more quickly than predicted by classical mechanics. This is because the object’s mass effectively increases as it approaches light speed, requiring exponentially more energy to accelerate. \\

\textbf{Length Contraction} \\
In special relativity, an object moving at a significant fraction of the speed of light appears shorter in the direction of motion when observed from a stationary frame. This effect becomes more pronounced as the object’s speed increases. \\


\textbf{Four-Momentum} \\
Four-momentum is an extension of the classical concept of momentum into four-dimensional spacetime in special relativity. It includes both the energy and the momentum of an object, and it remains conserved in relativistic interactions. \\

\textbf{Relativistic Mass} \\
In special relativity, the mass of an object increases as its velocity approaches the speed of light. This "relativistic mass" depends on the object's velocity and makes it harder to accelerate the object as it gets closer to light speed. \\


\section*{Wave Physics and Photoelectric Effect}

\textbf{Photon Energy} \\
The energy of a photon is directly related to its frequency. Higher frequency light, such as ultraviolet light, carries more energy than lower frequency light, such as infrared light. This is the foundation of quantum mechanics, where energy is quantized. \\


\textbf{Work Function} \\
The minimum energy required to eject an electron from the surface of a material when light shines on it. In the photoelectric effect, only photons with energy greater than the work function will release electrons from the material. \\


\section*{Quantum Mechanics}

\textbf{Schrödinger's Equation} \\
A fundamental equation in quantum mechanics that describes how the quantum state of a system evolves over time. The solution to this equation, called the wave function, contains all the information about the system. \\


\textbf{Wave Function} \\
In quantum mechanics, the wave function is a mathematical description of the probability amplitude of a particle's position and momentum. It provides a complete description of the quantum state of a system. The square of the wave function gives the probability of finding the particle at a specific position. \\

\textbf{Probability Distribution (Quantum Mechanics)} \\
The probability distribution in quantum mechanics gives the likelihood of finding a particle at a certain position in space. The square of the wave function \( |\Psi(x)|^2 \) gives the probability density, and it is normalized so that the total probability equals 1. \\

\textbf{Expectation Value} \\
The expectation value in quantum mechanics is the average value of a physical observable (such as position or momentum) over many measurements. It represents the center of the probability distribution described by the wave function. \\

\textbf{Uncertainty (Quantum Mechanics)} \\
The uncertainty principle in quantum mechanics states that certain pairs of physical properties, like position and momentum, cannot both be known to arbitrary precision. The more accurately one quantity is known, the less accurately the other can be measured. This is a fundamental concept in quantum mechanics. \\

\textbf{Operators and Eigenvalues} \\
In quantum mechanics, physical observables (such as energy, position, and momentum) are represented by operators. When an operator acts on a wave function and returns a constant multiple of that wave function, the wave function is called an eigenfunction, and the constant is the eigenvalue. \\

\textbf{Orthogonality} \\
In quantum mechanics, two wavefunctions are said to be orthogonal if their inner product is zero. This means that the probability of transitioning from one state to the other is zero.It implies that the two states are independent and cannot interfere with each other in quantum systems.

\section*{Relativistic Energy and Momentum}

\textbf{Rest Energy} \\
This concept, made famous by Einstein’s equation \( E = mc^2 \), states that any object with mass has an intrinsic energy, even if it is not moving. This energy is due to the object’s mass alone. \\


\textbf{Energy-Momentum Relation} \\
In relativity, energy and momentum are related in a more complex way than in classical mechanics. Even objects at rest have energy due to their mass, and the total energy of an object includes both its rest energy and the energy due to its motion (momentum). \\

\section*{Wave Function}

\textbf{Wave Function Solution} \\
The wave function describes the quantum state of a particle confined in a one-dimensional box with impenetrable walls. It takes the form of a sine function, where the quantum number \(n\) determines the allowed energy states. \\

\section*{Energy Levels}

\textbf{Quantized Energy Levels} \\
The energy of a particle in a box is quantized, meaning it can only take specific discrete values based on the quantum number \(n\). The lowest energy level corresponds to \(n=1\), and the energy increases with \(n^2\). \\

\textbf{Energy Difference} \\
The energy difference between two levels, such as between the ground state and the first excited state, is given by the difference in their energy values, where \(E_2 - E_1\) is particularly important for understanding transitions between states. \\

\section*{Uncertainty Principle}

\textbf{Heisenberg's Uncertainty Principle} \\
This principle asserts that it is impossible to simultaneously know the exact position and momentum of a particle. The more precisely one quantity is known, the less precise the other can be. This uncertainty is a fundamental aspect of quantum mechanics. \\


\section*{Hilbert Space and Inner Product}

\textbf{Hilbert Space} \\
A Hilbert space is a complete inner-product space that provides the mathematical foundation for quantum mechanics. It is the space where wave functions reside, and all physical states in quantum mechanics are vectors in this space. \\

\textbf{Inner Product (Quantum Mechanics)} \\
The inner product in quantum mechanics is a way to measure the overlap between two wave functions. It provides information about probabilities, expectation values, and the orthogonality of quantum states. The inner product is a generalization of the dot product used in vector spaces. \\

\section*{General Concepts}

\textbf{Superposition Principle} \\
In quantum mechanics, the superposition principle states that a quantum system can exist in multiple states simultaneously. The overall state is a combination (superposition) of these individual states. Superposition is a key concept in understanding quantum interference and entanglement. \\

\textbf{Normalization (Quantum Mechanics)} \\
Normalization is the process of adjusting the wave function of a quantum system so that the total probability of finding the particle somewhere in space equals one. This ensures that the wave function accurately represents a probability distribution. \\

\textbf{Hermitian operator} (or \textbf{self-adjoint operator}):

A linear operator \( \hat{A} \) that satisfies the condition
\[
\langle \phi, \hat{A} \psi \rangle = \langle \hat{A} \phi, \psi \rangle
\]
for all functions \( \phi \) and \( \psi \) in the Hilbert space. This means that the operator is invariant under complex conjugation and transposition. Hermitian operators have real eigenvalues, which correspond to the measurable values of an observable.
\\

\textbf{Even/odd function}
\begin{itemize}
        \item A function \(f(x)\) is an \textbf{even function} if:
        \[
        f(-x) = f(x).
        \]
        Even functions are symmetric about the \(y\)-axis.
    \[
    \int_{-\pi}^\pi f(x) \, dx = 2 \int_0^\pi f(x) \, dx.
    \]
        
        \item A function \(f(x)\) is an \textbf{odd function} if:
        \[
        f(-x) = -f(x).
        \]
        Odd functions are symmetric about the origin (and change sign for \(-x\)).
    \[
    \int_{-\pi}^\pi f(x) \, dx = 0.
    \]
    \end{itemize}



\end{multicols}

\end{footnotesize}

\end{document}
