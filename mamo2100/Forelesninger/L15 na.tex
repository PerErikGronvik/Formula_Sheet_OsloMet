\documentclass[a4paper,12pt]{article}
\usepackage{multicol}
\usepackage{geometry}
\usepackage{amsmath, amssymb}
\usepackage{titlesec} % To customize section title size and spacing
\geometry{a4paper, left=0.5in, right=0.5in, top=0.5in, bottom=0.5in}
\setlength{\parindent}{0pt} % Remove paragraph indentation
\setlength{\columnsep}{1cm} % Set spacing between columns

% Adjust section title font size and spacing
\titleformat{\section}
  {\normalfont\normalsize\bfseries} % Set section title font to small and bold
  {}{0pt}{}
\titlespacing*{\section}{0pt}{0.3cm}{0.2cm} % Adjust space before and after section

\begin{document}
\title{\textbf{L15 - 3D QM and Hydrogen Atom}}
\author{}
\date{}
\maketitle
\begin{multicols}{2}

\section*{3D Particle in a Box}
\subsection*{Problem Setup}
Consider a particle confined within a cubical box of side length \( L \). The potential \( U(x,y,z) \) is defined as:
\[
U(x,y,z) =
\begin{cases} 
0 & \text{if } 0 \leq x, y, z \leq L, \\
\infty & \text{otherwise}.
\end{cases}
\]
The particle is restricted to move only inside the box, where the potential is zero.

\subsection*{Schrödinger Equation (SE) in 3D}
The time-independent Schrödinger equation is:
\[
-\frac{\hbar^2}{2\mu} \nabla^2 \Psi(x,y,z) + U(x,y,z)\Psi(x,y,z) = 
\]
\[
E \Psi(x,y,z),
\]
where:
\begin{itemize}
    \item \( \nabla^2 = \frac{\partial^2}{\partial x^2} + \frac{\partial^2}{\partial y^2} + \frac{\partial^2}{\partial z^2} \) is the Laplace operator.
    \item \( \mu \) is the particle's mass.
    \item \( E \) is the energy of the particle.
\end{itemize}

\subsection*{Separation of Variables}
We assume the wavefunction can be separated as:
\[
\Psi(x,y,z) = X(x) Y(y) Z(z),
\]
and substitute this ansatz into the Schrödinger equation:
\[
-\frac{\hbar^2}{2\mu} \left[ \frac{1}{X} \frac{\mathrm{d}^2 X}{\mathrm{d}x^2} + \frac{1}{Y} \frac{\mathrm{d}^2 Y}{\mathrm{d}y^2} + \frac{1}{Z} \frac{\mathrm{d}^2 Z}{\mathrm{d}z^2} \right] = E.
\]
Separating the terms gives three independent equations:
\[
-\frac{\hbar^2}{2\mu} \frac{\mathrm{d}^2 X}{\mathrm{d}x^2} = E^x X, 
\]
\[
-\frac{\hbar^2}{2\mu} \frac{\mathrm{d}^2 Y}{\mathrm{d}y^2} = E^y Y, 
\]
\[
-\frac{\hbar^2}{2\mu} \frac{\mathrm{d}^2 Z}{\mathrm{d}z^2} = E^z Z,
\]
with \( E = E^x + E^y + E^z \).

\subsection*{Solutions to the 1D Equation}
Each of these equations has the boundary conditions \( X(0) = X(L) = 0 \) (and similarly for \( Y \) and \( Z \)). The solutions are:
\[
X_n(x) = \sqrt{\frac{2}{L}} \sin\left(\frac{n\pi x}{L}\right), 
\]
\[
Y_m(y) = \sqrt{\frac{2}{L}} \sin\left(\frac{m\pi y}{L}\right), 
\]
\[
Z_k(z) = \sqrt{\frac{2}{L}} \sin\left(\frac{k\pi z}{L}\right),
\]
where \( n, m, k \) are positive integers.

\subsection*{Energy Levels}
The corresponding energies are:
\[
E^x_n = \frac{\hbar^2}{2\mu} \left(\frac{n\pi}{L}\right)^2, 
\]
\[
E^y_m = \frac{\hbar^2}{2\mu} \left(\frac{m\pi}{L}\right)^2, 
\]
\[
E^z_k = \frac{\hbar^2}{2\mu} \left(\frac{k\pi}{L}\right)^2.
\]
The total energy is:
\[
E_{nmk} = \frac{\pi^2 \hbar^2}{2\mu L^2} (n^2 + m^2 + k^2).
\]

\subsection*{Wavefunctions}
The normalized wavefunction for the 3D box is:
\[
\Psi_{nmk}(x,y,z) = 
\]
\[
\left(\frac{2}{L}\right)^{3/2} 
\]
\[
\sin\left(\frac{n\pi x}{L}\right) 
\]
\[
\sin\left(\frac{m\pi y}{L}\right) 
\]
\[
\sin\left(\frac{k\pi z}{L}\right).
\]

\subsection*{Degeneracy}
The degeneracy of an energy level \( E_{nmk} \) depends on the number of distinct permutations of \( n^2 + m^2 + k^2 \) that yield the same value.

\section*{Hydrogen Atom in Quantum Mechanics}
\subsection*{Spherical Coordinates}
The hydrogen atom problem is best solved in spherical coordinates \( (r, \theta, \phi) \), where:
\[
r = \sqrt{x^2 + y^2 + z^2},
\]
\[
\theta = \text{zenith angle},
\]
\[
\phi = \text{azimuthal angle}.
\]

\subsection*{Schrödinger Equation}
The potential energy is:
\[
U(r) = -\frac{e^2}{4\pi \epsilon_0 r}.
\]
The Schrödinger equation becomes:
\[
-\frac{\hbar^2}{2\mu} \nabla^2 \Psi - \frac{e^2}{4\pi \epsilon_0 r} \Psi = E \Psi.
\]

\subsection*{Separation of Variables}
We assume \( \Psi(r,\theta,\phi) = R(r) Y(\theta, \phi) \), where \( Y(\theta, \phi) \) are spherical harmonics and \( R(r) \) is the radial solution.

\subsection*{Energy Levels}
Solving the radial equation leads to the quantized energy levels:
\[
E_n = -\frac{\mu e^4}{2 \hbar^2 (4\pi \epsilon_0)^2} \frac{1}{n^2}, \quad n = 1, 2, 3, \ldots
\]
In eV, the ground state energy is \( E_1 = -13.6 \, \mathrm{eV} \).

\subsection*{Quantum Numbers}
The wavefunction is characterized by three quantum numbers:
\begin{itemize}
    \item Principal quantum number \( n = 1, 2, 3, \ldots \),
    \item Orbital angular momentum quantum number \( l = 0, 1, \ldots, n-1 \),
    \item Magnetic quantum number \( m_l = -l, -(l-1), \ldots, l-1, l \).
\end{itemize}

\subsection*{Spherical Harmonics}
The angular part \( Y_{lm}(\theta, \phi) \) satisfies:
\[
\hat{L}^2 Y_{lm} = \hbar^2 l(l+1) Y_{lm}, \quad \hat{L}_z Y_{lm} = \hbar m Y_{lm}.
\]

\section*{Energy Levels in a 3D Box}
\subsection*{Problem Setup}
We consider a particle of mass \( \mu \) confined in a cubical box with sides of length \( L = \pi \). The particle experiences zero potential inside the box and infinite potential at the boundaries. The energy levels are given by:
\[
E_{n_x, n_y, n_z} = \frac{\hbar^2 \pi^2}{2\mu L^2} \left(n_x^2 + n_y^2 + n_z^2 \right),
\]
where \( n_x, n_y, n_z \) are positive integers representing quantum numbers along the \( x, y, z \)-axes.

\subsection*{Ground State Energy}
The ground state corresponds to \( n_x = n_y = n_z = 1 \). Substituting these values:
\[
E_\text{gs} = \frac{\hbar^2 \pi^2}{2\mu L^2} \left(1^2 + 1^2 + 1^2\right).
\]
With \( L = \pi \), the expression simplifies:
\[
E_\text{gs} = \frac{\hbar^2}{2\mu} \cdot 3.
\]
Thus, the ground-state energy is:
\[
E_\text{gs} = \frac{3\hbar^2}{2\mu}.
\]

\subsection*{Degeneracy of Energy Levels}
For \( E = \frac{11\hbar^2}{2\mu} \), the quantum numbers must satisfy:
\[
n_x^2 + n_y^2 + n_z^2 = 11.
\]
The possible combinations of \( (n_x, n_y, n_z) \) are:
\[
(3, 1, 1), \quad (1, 3, 1), \quad (1, 1, 3).
\]
Thus, the energy level \( E = \frac{11\hbar^2}{2\mu} \) is 3-fold degenerate.

\section*{Inductive Proof of Integral Formula}
\subsection*{Problem Setup}
We aim to prove, using induction and integration by parts, the formula:
\[
\int_0^\infty r^n e^{-ar} \, \mathrm{d}r = \frac{n!}{a^{n+1}}, \quad a > 0.
\]

\subsection*{Base Case: \( n = 0 \)}
When \( n = 0 \), the integral reduces to:
\[
\int_0^\infty e^{-ar} \, \mathrm{d}r = \left[-\frac{1}{a} e^{-ar} \right]_0^\infty = \frac{1}{a}.
\]
Since \( 0! = 1 \), the base case holds:
\[
\int_0^\infty e^{-ar} \, \mathrm{d}r = \frac{0!}{a^{0+1}}.
\]

\subsection*{Inductive Step}
Assume the formula holds for \( n = k \):
\[
\int_0^\infty r^k e^{-ar} \, \mathrm{d}r = \frac{k!}{a^{k+1}}.
\]
We prove it for \( n = k+1 \) using integration by parts. Let:
\[
u = r^{k+1}, \quad \mathrm{d}v = e^{-ar} \, \mathrm{d}r.
\]
Then:
\[
\mathrm{d}u = (k+1)r^k \, \mathrm{d}r, \quad v = -\frac{1}{a} e^{-ar}.
\]
By the integration by parts formula \( \int u \, \mathrm{d}v = uv - \int v \, \mathrm{d}u \), we get:
\[
\int_0^\infty r^{k+1} e^{-ar} \, \mathrm{d}r =
\]
\[
\left[-\frac{r^{k+1}}{a} e^{-ar} \right]_0^\infty + \frac{k+1}{a} \int_0^\infty r^k e^{-ar} \, \mathrm{d}r.
\]
The boundary term vanishes because:
\[
\left[-\frac{r^{k+1}}{a} e^{-ar} \right]_0^\infty = 
\]
\[
0 \quad \text{(as \( e^{-ar} \to 0 \) when \( r \to \infty \))}.
\]
Substitute the inductive hypothesis:
\[
\int_0^\infty r^{k+1} e^{-ar} \, \mathrm{d}r = \frac{k+1}{a} \cdot \frac{k!}{a^{k+1}} = \frac{(k+1)!}{a^{k+2}}.
\]
Thus, the formula holds for \( n = k+1 \), completing the proof by induction.

\section*{Wavefunction of Hydrogen Ground State}
\subsection*{Normalization of Wavefunction}
The wavefunction is given as:
\[
\psi(r) = \frac{1}{\sqrt{\pi}} e^{-r}.
\]
To verify normalization, compute:
\[
\int_0^\infty |\psi(r)|^2 4\pi r^2 \, \mathrm{d}r = 4\pi \int_0^\infty \left(\frac{1}{\sqrt{\pi}} e^{-r}\right)^2 r^2 \, \mathrm{d}r.
\]
Simplify:
\[
\int_0^\infty |\psi(r)|^2 4\pi r^2 \, \mathrm{d}r = \frac{4}{\pi} \int_0^\infty r^2 e^{-2r} \, \mathrm{d}r.
\]
Using the result from the integral formula with \( n=2 \) and \( a=2 \):
\[
\int_0^\infty r^2 e^{-2r} \, \mathrm{d}r = \frac{2!}{2^3} = \frac{2}{8} = \frac{1}{4}.
\]
Substitute:
\[
\int_0^\infty |\psi(r)|^2 4\pi r^2 \, \mathrm{d}r = \frac{4}{\pi} \cdot \frac{1}{4} \cdot \pi = 1.
\]
Thus, the wavefunction is normalized.

\subsection*{Energy of Ground State}
The Schrödinger equation for the radial wavefunction is:
\[
-\frac{1}{2} \psi''(r) - \frac{1}{r} \psi'(r) - \frac{1}{r} \psi(r) = E_n \psi(r).
\]
Substitute \( \psi(r) = \frac{1}{\sqrt{\pi}} e^{-r} \):
\[
\psi'(r) = -\frac{1}{\sqrt{\pi}} e^{-r}, \quad \psi''(r) = \frac{1}{\sqrt{\pi}} e^{-r}.
\]
Simplify:
\[
-\frac{1}{2} \psi''(r) - \frac{1}{r} \psi'(r) - \frac{1}{r} \psi(r) = 
\]
\[
-\frac{1}{2} \frac{1}{\sqrt{\pi}} e^{-r} + \frac{1}{r} \frac{1}{\sqrt{\pi}} e^{-r} - \frac{1}{r} \frac{1}{\sqrt{\pi}} e^{-r}.
\]
Notice terms cancel, leaving:
\[
-\frac{1}{2} \frac{1}{\sqrt{\pi}} e^{-r} = E_1 \frac{1}{\sqrt{\pi}} e^{-r}.
\]
Thus:
\[
E_1 = -\frac{1}{2}.
\]

\end{multicols}
\end{document}
