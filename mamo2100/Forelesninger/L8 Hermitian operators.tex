\documentclass[a4paper,12pt]{article}
\usepackage{multicol}
\usepackage{geometry}
\usepackage{amsmath, amssymb}
\usepackage{titlesec} % To customize section title size and spacing
\geometry{a4paper, left=0.5in, right=0.5in, top=0.5in, bottom=0.5in}
\setlength{\parindent}{0pt} % Remove paragraph indentation
\setlength{\columnsep}{1cm} % Set spacing between columns

% Adjust section title font size and spacing
\titleformat{\section}
  {\normalfont\normalsize\bfseries} % Set section title font to small and bold
  {}{0pt}{}
\titlespacing*{\section}{0pt}{0.3cm}{0.2cm} % Adjust space before and after section

\begin{document}
\title{L8 - QM Hermitian operators}
\author{}
\date{}
\maketitle

\begin{multicols}{2}

\section*{Hermitian Operators and Their Properties}

In quantum mechanics, operators corresponding to physical observables (e.g., position $\hat{x}$, momentum $\hat{p}$, and energy $\hat{H}$) must be Hermitian. This ensures that their eigenvalues, representing measurable quantities, are real.

\subsection*{Definition of a Hermitian Operator}
An operator $\hat{A}$ is Hermitian if:
\[
\langle \phi, \hat{A} \psi \rangle = \langle \hat{A} \phi, \psi \rangle \quad \forall \, \phi, \psi \in V
\]
where $V$ is a Hilbert space. In terms of integrals:
\[
\int_{-\infty}^\infty \phi^*(x) (\hat{A} \psi(x)) dx = \int_{-\infty}^\infty (\hat{A} \phi(x))^* \psi(x) dx
\]

For the real case:
\[
\langle \psi, \hat{A} \psi \rangle = \langle \hat{A} \psi, \psi \rangle \implies \langle \psi, \hat{A} \psi \rangle \in \mathbb{R}.
\]

\subsection*{Proposition: Real Eigenvalues of Hermitian Operators}
If $\hat{A}$ is Hermitian, then $\langle \psi, \hat{A} \psi \rangle$ is real.

\textbf{Proof:}
\[
\langle A \rangle = \langle \psi, \hat{A} \psi \rangle.
\]
Since $\hat{A}$ is Hermitian:
\[
\langle A \rangle = \langle \psi, \hat{A} \psi \rangle = \langle \hat{A} \psi, \psi \rangle = \langle A \rangle^*.
\]
Thus, $\langle A \rangle \in \mathbb{R}$.

\section*{Cauchy-Schwarz Inequality in Quantum Mechanics}

\subsection*{Inner Product and Norms}
The inner product in quantum mechanics is defined as:
\[
\langle \phi, \psi \rangle = \int_{-\infty}^\infty \phi^*(x) \psi(x) dx.
\]
The norm associated with this inner product is:
\[
\|\psi\| = \sqrt{\langle \psi, \psi \rangle} = \sqrt{\int_{-\infty}^\infty |\psi(x)|^2 dx}.
\]

Normalization of wavefunctions requires:
\[
\|\psi\| = 1 \implies \int_{-\infty}^\infty |\psi(x)|^2 dx = 1.
\]

\subsection*{Cauchy-Schwarz Inequality}
For any two functions $\phi(x)$ and $\psi(x)$:
\[
|\langle \phi, \psi \rangle|^2 \leq \|\phi\|^2 \|\psi\|^2.
\]

\textbf{Proof:}
Expand $\|\phi + \lambda \psi\|^2 \geq 0$, where $\lambda$ is a complex number, and simplify to derive the inequality.

\section*{Uncertainty in Quantum Mechanics}

\subsection*{Uncertainty Definition}
The uncertainty in the measurement of an operator $\hat{A}$ is:
\[
\Delta A = \sqrt{\langle \hat{A}^2 \rangle - \langle \hat{A} \rangle^2}.
\]

\subsection*{Uncertainty Relation for Hermitian Operators}
For any Hermitian operator $\hat{A}$:
\[
\langle \hat{A}^2 \rangle \geq \langle \hat{A} \rangle^2 \implies (\Delta A)^2 \geq 0.
\]

\textbf{Proof:}
\[
\langle A \rangle^2 = \left( \int_{-\infty}^\infty \psi^*(x) (\hat{A} \psi(x)) dx \right)^2.
\]
Apply the Cauchy-Schwarz inequality:
\[
\langle A \rangle^2 \leq \int |\psi(x)|^2 dx \int |(\hat{A} \psi(x))|^2 dx = \langle \psi, \hat{A}^2 \psi \rangle.
\]
Thus:
\[
\langle \hat{A}^2 \rangle - \langle \hat{A} \rangle^2 \geq 0.
\]

\section*{Momentum and Position Operators}

\subsection*{Momentum Operator is Hermitian}
The momentum operator $\hat{p} = -i\hbar \frac{d}{dx}$ is Hermitian if boundary conditions ensure:
\[
\int_{-\infty}^\infty \phi^*(x) \left(-i\hbar \frac{d\psi}{dx}\right) dx = \int_{-\infty}^\infty \left(i\hbar \frac{d\phi^*}{dx}\right) \psi(x) dx.
\]

Boundary terms vanish if $\psi(x) \to 0$ as $x \to \pm\infty$. This ensures:
\[
\langle \phi, \hat{p} \psi \rangle = \langle \hat{p} \phi, \psi \rangle.
\]

\subsection*{Energy Operator}
The kinetic energy operator $\hat{T}$ is:
\[
\hat{T} = \frac{\hat{p}^2}{2m} = -\frac{\hbar^2}{2m} \frac{d^2}{dx^2}.
\]
Since $\hat{p}$ is Hermitian, $\hat{T}$ is also Hermitian, as shown through integration by parts.

\section*{Wavefunctions in Hilbert Space}
Wavefunctions $\psi(x)$ live in a Hilbert space, which is a complete inner-product space. Completeness ensures all Cauchy sequences converge.

\subsection*{Example: Position Expectation}
For the position operator $\hat{x}$:
\[
\langle x \rangle = \int_{-\infty}^\infty x |\psi(x)|^2 dx.
\]

The variance in position is:
\[
(\Delta x)^2 = \langle x^2 \rangle - \langle x \rangle^2.
\]

\end{multicols}
\section*{Momentum Operator Expectation for Real-Valued Wavefunctions}

\subsection*{Goal}
Show that for a real-valued wavefunction $\psi(x)$, the expectation value of the momentum operator $\hat{p} = -i\hbar \frac{d}{dx}$ is zero:
\[
\langle \hat{p} \rangle = \langle \psi, \hat{p} \psi \rangle = 0.
\]

\subsection*{Solution}
The expectation value of $\hat{p}$ is defined as:
\[
\langle \hat{p} \rangle = \int_{-\infty}^\infty \psi^*(x) \left(-i\hbar \frac{d\psi(x)}{dx} \right) dx.
\]

Since $\psi(x)$ is real-valued:
\[
\psi^*(x) = \psi(x).
\]

Thus, the expression simplifies to:
\[
\langle \hat{p} \rangle = -i\hbar \int_{-\infty}^\infty \psi(x) \frac{d\psi(x)}{dx} dx.
\]

Now let:
\[
I = \int_{-\infty}^\infty \psi(x) \frac{d\psi(x)}{dx} dx.
\]

\subsection*{Simplifying $I$}
Using the product rule for derivatives:
\[
\frac{d}{dx} \left(\psi^2(x)\right) = 2\psi(x) \frac{d\psi(x)}{dx}.
\]

Rewriting:
\[
\psi(x) \frac{d\psi(x)}{dx} = \frac{1}{2} \frac{d}{dx} \left(\psi^2(x)\right).
\]

Substituting back into $I$:
\[
I = \frac{1}{2} \int_{-\infty}^\infty \frac{d}{dx} \left(\psi^2(x)\right) dx.
\]

Using the fundamental theorem of calculus:
\[
\int_{-\infty}^\infty \frac{d}{dx} \left(\psi^2(x)\right) dx = \left[\psi^2(x)\right]_{-\infty}^\infty.
\]

Since $\psi(x) \to 0$ as $x \to \pm \infty$ (normalization condition), this boundary term vanishes:
\[
\left[\psi^2(x)\right]_{-\infty}^\infty = 0.
\]

Thus:
\[
I = 0 \implies \langle \hat{p} \rangle = 0.
\]

\subsection*{Conclusion}
For real-valued wavefunctions, the expectation value of the momentum operator $\hat{p}$ is zero:
\[
\boxed{\langle \hat{p} \rangle = 0.}
\]

\section*{Kinetic Energy Operator is Hermitian}

\subsection*{Goal}
Show that the kinetic energy operator:
\[
\hat{T} = \frac{\hat{p}^2}{2m} = -\frac{\hbar^2}{2m} \frac{d^2}{dx^2}
\]
is Hermitian.

\subsection*{Definition of Hermitian Operator}
An operator $\hat{A}$ is Hermitian if:
\[
\langle \phi, \hat{A} \psi \rangle = \langle \hat{A} \phi, \psi \rangle \quad \forall \phi, \psi.
\]

\subsection*{Solution: Proof for $\hat{T}$}
Start with the left-hand side:
\[
\langle \phi, \hat{T} \psi \rangle = \int_{-\infty}^\infty \phi^*(x) \left(-\frac{\hbar^2}{2m} \frac{d^2\psi(x)}{dx^2}\right) dx.
\]

Simplify:
\[
\langle \phi, \hat{T} \psi \rangle = -\frac{\hbar^2}{2m} \int_{-\infty}^\infty \phi^*(x) \frac{d^2\psi(x)}{dx^2} dx.
\]

To simplify further, apply integration by parts twice.

\subsection*{First Integration by Parts}
Let:
\[
u = \phi^*(x), \quad dv = \frac{d^2\psi(x)}{dx^2} dx.
\]

Then:
\[
du = \frac{d\phi^*(x)}{dx} dx, \quad v = \frac{d\psi(x)}{dx}.
\]

Integration by parts gives:
\[
\int_{-\infty}^\infty \phi^*(x) \frac{d^2\psi(x)}{dx^2} dx = \left[\phi^*(x) \frac{d\psi(x)}{dx}\right]_{-\infty}^\infty - \int_{-\infty}^\infty \frac{d\phi^*(x)}{dx} \frac{d\psi(x)}{dx} dx.
\]

The boundary term vanishes because $\psi(x) \to 0$ as $x \to \pm \infty$:
\[
\left[\phi^*(x) \frac{d\psi(x)}{dx}\right]_{-\infty}^\infty = 0.
\]

Thus:
\[
\int_{-\infty}^\infty \phi^*(x) \frac{d^2\psi(x)}{dx^2} dx = -\int_{-\infty}^\infty \frac{d\phi^*(x)}{dx} \frac{d\psi(x)}{dx} dx.
\]

\subsection*{Second Integration by Parts}
Apply integration by parts again to:
\[
\int_{-\infty}^\infty \frac{d\phi^*(x)}{dx} \frac{d\psi(x)}{dx} dx.
\]

Let:
\[
u = \frac{d\phi^*(x)}{dx}, \quad dv = \frac{d\psi(x)}{dx} dx.
\]

Then:
\[
du = \frac{d^2\phi^*(x)}{dx^2} dx, \quad v = \psi(x).
\]

Integration by parts gives:
\[
\int_{-\infty}^\infty \frac{d\phi^*(x)}{dx} \frac{d\psi(x)}{dx} dx = \left[\frac{d\phi^*(x)}{dx} \psi(x)\right]_{-\infty}^\infty - \int_{-\infty}^\infty \phi^*(x) \frac{d^2\psi(x)}{dx^2} dx.
\]

Again, the boundary term vanishes:
\[
\left[\frac{d\phi^*(x)}{dx} \psi(x)\right]_{-\infty}^\infty = 0.
\]

Thus:
\[
\int_{-\infty}^\infty \frac{d\phi^*(x)}{dx} \frac{d\psi(x)}{dx} dx = -\int_{-\infty}^\infty \phi^*(x) \frac{d^2\psi(x)}{dx^2} dx.
\]

\subsection*{Final Expression}
Substitute back into the original integral:
\[
\langle \phi, \hat{T} \psi \rangle = \int_{-\infty}^\infty \left(-\frac{\hbar^2}{2m} \phi^*(x) \frac{d^2\psi(x)}{dx^2}\right) dx.
\]

Since the same result holds for $\langle \hat{T} \phi, \psi \rangle$, $\hat{T}$ is Hermitian.

\subsection*{Conclusion}
The kinetic energy operator $\hat{T}$ is Hermitian:
\[
\boxed{\hat{T} \text{ is Hermitian.}}
\]
\end{document}
