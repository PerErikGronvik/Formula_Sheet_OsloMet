\documentclass[a4paper,12pt]{article}
\usepackage{multicol}
\usepackage{geometry}
\usepackage{amsmath}
\usepackage{titlesec}
\geometry{a4paper, left=0.5in, right=0.5in, top=0.5in, bottom=0.5in}
\setlength{\parindent}{0pt}
\setlength{\columnsep}{1cm}

\titleformat{\section}
  {\normalfont\normalsize\bfseries}
  {}{0pt}{}
\titlespacing*{\section}{0pt}{0.3cm}{0.2cm}

\begin{document}

\title{L5 - Relativity Notes: Work, Energy, and Mass}
\author{}
\date{}
\maketitle

\begin{multicols}{2}

\section{Work Done to Accelerate a Particle}
We aim to calculate the work \( W \) required to accelerate a particle of mass \( m \) from speed \( v_1 = 0.8c \) to \( v_2 = 0.9c \). The relativistic work is expressed as:
\[
W = \int_{v_1}^{v_2} \frac{mv \, dv}{(1 - v^2/c^2)^{3/2}}.
\]

\subsection*{Step 1: Simplify the Integral}
Define a substitution:
\[
\omega = 1 - \frac{v^2}{c^2}, \quad \text{which implies} \quad d\omega = -\frac{2v}{c^2} \, dv.
\]
Rewriting \( v \, dv \) using this substitution:
\[
v \, dv = -\frac{c^2}{2} \, d\omega.
\]

Update the limits of integration:
\[
\omega_1 = 1 - \frac{v_1^2}{c^2}, \quad \omega_2 = 1 - \frac{v_2^2}{c^2}.
\]

\subsection*{Step 2: Rewrite the Integral in Terms of \( \omega \)}
The work equation becomes:
\[
W = \int_{\omega_1}^{\omega_2} \frac{-mc^2/2}{\omega^{3/2}} \, d\omega.
\]
Simplify:
\[
W = -\frac{mc^2}{2} \int_{\omega_1}^{\omega_2} \omega^{-3/2} \, d\omega.
\]

\subsection*{Step 3: Solve the Integral}
The integral of \( \omega^{-3/2} \) is:
\[
\int \omega^{-3/2} \, d\omega = -2\omega^{-1/2}.
\]
Apply this to compute:
\[
W = -\frac{mc^2}{2} \Big[ -2\omega^{-1/2} \Big]_{\omega_1}^{\omega_2}.
\]
Simplify:
\[
W = mc^2 \Big[ \frac{1}{\sqrt{\omega_2}} - \frac{1}{\sqrt{\omega_1}} \Big].
\]

\subsection*{Step 4: Compute for \( v_1 = 0.8c \) and \( v_2 = 0.9c \)}
First, calculate \( \omega_1 \) and \( \omega_2 \):
\[
\omega_1 = 1 - \frac{(0.8c)^2}{c^2} = 1 - 0.64 = 0.36,
\]
\[
\omega_2 = 1 - \frac{(0.9c)^2}{c^2} = 1 - 0.81 = 0.19.
\]

Substitute into the work equation:
\[
W = mc^2 \Big[ \frac{1}{\sqrt{0.19}} - \frac{1}{\sqrt{0.36}} \Big].
\]

Evaluate:
\[
\sqrt{0.19} \approx 0.436, \quad \sqrt{0.36} = 0.6,
\]
\[
\frac{1}{\sqrt{0.19}} \approx 2.294, \quad \frac{1}{\sqrt{0.36}} = 1.667.
\]

Thus:
\[
W = mc^2 \Big[ 2.294 - 1.667 \Big] = mc^2 (0.627).
\]

\subsection*{Final Answer}
The work required is:
\[
W = 0.627 \, mc^2.
\]

\section{Relativistic Total Energy and Speed}
We now analyze a particle where its kinetic energy is five times its rest energy. The kinetic energy is related to the Lorentz factor \( \gamma \) as:
\[
K = (\gamma - 1)mc^2.
\]

\subsection*{Step 1: Solve for \( \gamma \)}
Given \( K = 5mc^2 \), we have:
\[
\gamma - 1 = 5 \quad \Rightarrow \quad \gamma = 6.
\]

The total energy is:
\[
E = \gamma mc^2 = 6mc^2.
\]

\subsection*{Step 2: Find the Speed \( v \)}
The Lorentz factor \( \gamma \) is related to \( v \) by:
\[
\gamma = \frac{1}{\sqrt{1 - v^2/c^2}}.
\]
Substitute \( \gamma = 6 \):
\[
6 = \frac{1}{\sqrt{1 - v^2/c^2}} \quad \Rightarrow \quad \sqrt{1 - v^2/c^2} = \frac{1}{6}.
\]

Square both sides:
\[
1 - v^2/c^2 = \frac{1}{36}.
\]

Rearrange:
\[
v^2/c^2 = 1 - \frac{1}{36} = \frac{35}{36}.
\]

Thus:
\[
v = c\sqrt{\frac{35}{36}} = c\frac{\sqrt{35}}{6}.
\]

\subsection*{Final Answer}
The speed is:
\[
v = c\frac{\sqrt{35}}{6}.
\]

\section{Derivation of Energy-Momentum Relation}
We now prove that the total energy of a particle is given by:
\[
E^2 = (pc)^2 + (mc^2)^2.
\]

\subsection*{Step 1: Define Momentum and Energy}
The relativistic momentum is:
\[
p = \gamma mv \quad \Rightarrow \quad p^2 = \frac{m^2v^2}{1 - v^2/c^2}.
\]

The total energy is:
\[
E = \gamma mc^2 \quad \Rightarrow \quad E^2 = \frac{m^2c^4}{1 - v^2/c^2}.
\]

\subsection*{Step 2: Combine Equations}
Factorize \( E^2 - p^2c^2 \):
\[
E^2 - p^2c^2 = \frac{m^2c^4}{1 - v^2/c^2} - \frac{m^2v^2c^2}{1 - v^2/c^2}.
\]

Simplify:
\[
E^2 - p^2c^2 = \frac{m^2c^4 - m^2v^2c^2}{1 - v^2/c^2}.
\]

Factorize the numerator:
\[
E^2 - p^2c^2 = \frac{m^2c^2(c^2 - v^2)}{1 - v^2/c^2}.
\]

Simplify further:
\[
E^2 - p^2c^2 = m^2c^2.
\]

Thus:
\[
E^2 = (pc)^2 + (mc^2)^2.
\]

\subsection*{Final Answer}
The energy-momentum relation is:
\[
E^2 = (pc)^2 + (mc^2)^2.
\]

\section{Derivation of Energy-Momentum Relation}
The energy-momentum relation for a particle of rest mass $m$ is given by:
\[
E^2 = (pc)^2 + (mc^2)^2.
\]

\subsection*{Derivation}
The total energy of a particle is expressed as:
\[
E = \gamma mc^2,
\]
where $\gamma = \frac{1}{\sqrt{1 - v^2/c^2}}$. Substituting $\gamma$:
\[
E = \frac{mc^2}{\sqrt{1 - v^2/c^2}}. \tag{L1}
\]

Squaring both sides:
\[
E^2 = \frac{(mc^2)^2}{1 - v^2/c^2}.
\]

Rearranging terms to separate the rest energy:
\[
E^2 = \frac{v^2}{c^2}E^2 + (mc^2)^2.
\]

Let $p = \gamma mv$ be the relativistic momentum. Then $E = pc^2/v$, and substituting into the expression above:
\[
E^2 = (pc)^2 + (mc^2)^2. \tag{L3}
\]

This confirms the energy-momentum relation.

---

\section{Calculating Work to Accelerate a Particle}
The work done to accelerate a particle of mass $m$ from speed $v_1$ to $v_2$ is:
\[
W = \int_{v_1}^{v_2} \frac{mv \, dv}{(1 - v^2/c^2)^{3/2}}.
\]

\subsection*{Setup and Integration}
Given $v_1 = 0.8c$ and $v_2 = 0.9c$, we substitute:
\[
W = \int_{0.8c}^{0.9c} \frac{mv \, dv}{(1 - v^2/c^2)^{3/2}}.
\]

Using substitution $u = 1 - v^2/c^2 \implies du = -\frac{2v}{c^2}dv$:
\[
W = -\frac{mc^2}{2} \int_{u_1}^{u_2} u^{-3/2} \, du,
\]
where $u_1 = 1 - (0.8)^2 = 0.36$ and $u_2 = 1 - (0.9)^2 = 0.19$.

Evaluating the integral:
\[
W = \frac{mc^2}{2} \left[ -2u^{-1/2} \right]_{0.36}^{0.19}.
\]

Simplify:
\[
W = mc^2 \left[ \frac{1}{\sqrt{0.19}} - \frac{1}{\sqrt{0.36}} \right].
\]

Numerical evaluation gives:
\[
W \approx 0.627mc^2.
\]

---

\section{Kinetic Energy and Speed of an Electron}
An electron has a kinetic energy five times its rest energy. Calculate its total energy $E$ and speed $v$.

\subsection*{Solution}
The total energy $E$ is the sum of rest energy $mc^2$ and kinetic energy $K$. Given $K = 5mc^2$, the total energy is:
\[
E = K + mc^2 = 5mc^2 + mc^2 = 6mc^2.
\]

To find $v$, we use the Lorentz factor:
\[
E = \gamma mc^2 \implies \gamma = \frac{E}{mc^2} = 6.
\]

Recall $\gamma = \frac{1}{\sqrt{1 - v^2/c^2}}$:
\[
6 = \frac{1}{\sqrt{1 - v^2/c^2}}.
\]

Squaring both sides:
\[
36 = \frac{1}{1 - v^2/c^2} \implies 1 - v^2/c^2 = \frac{1}{36}.
\]

Solving for $v^2/c^2$:
\[
v^2/c^2 = 1 - \frac{1}{36} = \frac{35}{36}.
\]

Thus:
\[
v = c \sqrt{\frac{35}{36}} = \frac{\sqrt{35}}{6}c.
\]

---

\section{Speed of a Particle with Increased Energy}
The total energy of a particle is 50\% greater than its rest energy. Find its speed $v$.

\subsection*{Solution}
The total energy is:
\[
E = 1.5mc^2.
\]

Using $\gamma$:
\[
E = \gamma mc^2 \implies \gamma = 1.5.
\]

From $\gamma = \frac{1}{\sqrt{1 - v^2/c^2}}$:
\[
1.5 = \frac{1}{\sqrt{1 - v^2/c^2}}.
\]

Squaring:
\[
2.25 = \frac{1}{1 - v^2/c^2} \implies 1 - v^2/c^2 = \frac{1}{2.25} = \frac{4}{9}.
\]

Thus:
\[
v^2/c^2 = 1 - \frac{4}{9} = \frac{5}{9}.
\]

Finally:
\[
v = c \sqrt{\frac{5}{9}} = \frac{\sqrt{5}}{3}c \approx 0.75c.
\]

\end{multicols}
\end{document}
