\documentclass[a4paper,12pt]{article}
\usepackage{multicol}
\usepackage{geometry}
\usepackage{amsmath}
\usepackage{amsfonts}
\usepackage{amssymb}
\usepackage{titlesec} % To customize section title size and spacing
\geometry{a4paper, left=0.5in, right=0.5in, top=0.5in, bottom=0.5in}
\setlength{\parindent}{0pt} % Remove paragraph indentation
\setlength{\columnsep}{1cm} % Set spacing between columns

% Adjust section title font size and spacing
\titleformat{\section}
  {\normalfont\normalsize\bfseries} % Set section title font to small and bold
  {}{0pt}{}
\titlespacing*{\section}{0pt}{0.3cm}{0.2cm} % Adjust space before and after section

\begin{document}
\title{L14 - Variational Principle and Quantum Harmonic Oscillator}
\author{}
\date{}
\maketitle
\begin{multicols}{2}

\section{Particle in a Box: Trial Wavefunction}

We begin by solving a standard quantum mechanics problem: the particle in a one-dimensional box with the potential
\[
U(x) = 
\begin{cases} 
0, & |x| \leq \frac{1}{2} \\ 
\infty, & |x| > \frac{1}{2}.
\end{cases}
\]

The wavefunction must vanish at the boundaries of the box, i.e., \( \psi\left(\pm \frac{1}{2}\right) = 0 \). A trial wavefunction is chosen:
\[
\psi(x) = C\left(x^2 - \frac{1}{4}\right),
\]
where \( C \) is a normalization constant. This wavefunction satisfies the boundary conditions and is admissible.

\subsection*{Ground-State Energy of the Particle in a Box}
The ground-state energy for the exact solution is given by:
\[
E_1 = \frac{\pi^2 \hbar^2}{2mL^2},
\]
where \( L = 1 \) in this case (the box width).

\subsection*{Energy Expectation Value of the Trial Wavefunction}
The expectation value of the energy for a trial wavefunction \( \psi(x) \) is given by:
\[
E[\psi] = \frac{\langle \psi | \hat{H} | \psi \rangle}{\langle \psi | \psi \rangle},
\]
where the Hamiltonian is \( \hat{H} = -\frac{\hbar^2}{2m}\frac{d^2}{dx^2} \). 

\subsubsection*{Step 1: Compute \( \langle \psi | \hat{H} | \psi \rangle \)}
\[
\langle \psi | \hat{H} | \psi \rangle = \int_{-\frac{1}{2}}^{\frac{1}{2}} \psi(x) \left(-\frac{\hbar^2}{2m} \frac{d^2}{dx^2}\psi(x)\right) dx.
\]
First, compute the derivatives of \( \psi(x) \):
\[
\psi'(x) = 2Cx, \quad \psi''(x) = 2C.
\]
Substitute into the integrand:
\[
\langle \psi | \hat{H} | \psi \rangle = \int_{-\frac{1}{2}}^{\frac{1}{2}} C\left(x^2 - \frac{1}{4}\right) \left(-\frac{\hbar^2}{2m} \cdot 2C\right) dx.
\]
Simplify:
\[
\langle \psi | \hat{H} | \psi \rangle = -\frac{\hbar^2 C^2}{m} \int_{-\frac{1}{2}}^{\frac{1}{2}} \left(x^2 - \frac{1}{4}\right) dx.
\]
Using symmetry, the integral over \( x^3 \) vanishes, leaving:
\[
\int_{-\frac{1}{2}}^{\frac{1}{2}} \left(x^2 - \frac{1}{4}\right) dx = \frac{2}{3}\cdot\frac{1}{8} - \frac{1}{4}.
\]

\subsubsection*{Step 2: Compute Normalization \( \langle \psi | \psi \rangle \)}
\[
\langle \psi | \psi \rangle = \int_{-\frac{1}{2}}^{\frac{1}{2}} \left(C^2 \left(x^2 - \frac{1}{4}\right)^2\right) dx.
\]

Compute term-by-term:
\[
\int_{-\frac{1}{2}}^{\frac{1}{2}} x^4 dx = \frac{1}{5},
\]
\[
\quad \int_{-\frac{1}{2}}^{\frac{1}{2}} x^2 \cdot \frac{1}{4} dx = 0,
\]
\[
\quad \int_{-\frac{1}{2}}^{\frac{1}{2}} \left(\frac{1}{4}\right)^2 dx = \frac{1}{16}.
\]

Substitute into \( \langle \psi | \psi \rangle \), normalize \( C \), and substitute into \( E[\psi] \). Verify that \( E[\psi] \geq E_1 \), as required by the variational principle.

\section{Variational Principle}
The variational principle states:
\[
\frac{\langle \psi | \hat{H} | \psi \rangle}{\langle \psi | \psi \rangle} \geq E_1,
\]
where \( E_1 \) is the ground-state energy, and equality holds only when \( \psi(x) \) is the exact ground-state wavefunction.

\section{Quantum Harmonic Oscillator (QHO)}

The quantum harmonic oscillator potential is:
\[
U(x) = \frac{1}{2}m\omega^2x^2,
\]
where \( m \) is the particle's mass and \( \omega \) is the angular frequency.

\subsection*{Schrödinger Equation for QHO}
The time-independent Schrödinger equation reads:
\[
-\frac{\hbar^2}{2m} \frac{d^2}{dx^2} \psi(x) + \frac{1}{2}m\omega^2x^2 \psi(x) = E \psi(x).
\]

\subsection*{Solutions to the QHO}
The solutions are of the form:
\[
\psi_n(x) = H_n(x) e^{-\frac{1}{2}\alpha x^2},
\]
where \( \alpha = \frac{m\omega}{\hbar} \) and \( H_n(x) \) are Hermite polynomials:
\[
H_0(x) = 1, \quad H_1(x) = 2x, \quad H_2(x) = 4x^2 - 2, \dots
\]

\subsection*{Energy Levels}
The energy levels of the QHO are quantized:
\[
E_n = \hbar\omega\left(n + \frac{1}{2}\right), \quad n = 0, 1, 2, \dots
\]
The ground-state energy is:
\[
E_0 = \frac{1}{2}\hbar\omega.
\]

\subsection*{Zero-Point Energy and Spacing}
The QHO has a zero-point energy of \( \frac{1}{2}\hbar\omega \), and the energy levels are equally spaced by \( \hbar\omega \).

\section{Numerical Solution of the Harmonic Oscillator (HO)}

\subsection*{Problem Statement}
We aim to solve the Schrödinger equation numerically for the harmonic oscillator (HO) potential:
\[
-\frac{1}{2}\psi''(x) + \frac{1}{2}x^2 \psi(x) = E \psi(x),
\]
to determine the ground-state wavefunction \( \psi_0(x) \) and energy \( E_0 \). The exact solutions are already known:
\[
E_0 = \frac{1}{2}, \quad \psi_0(x) = Ce^{-x^2/2}.
\]

\subsection*{Discretization of the Schrödinger Equation}
The potential \( V(x) = \frac{1}{2}x^2 \) is quadratic. To solve this equation numerically:
\begin{itemize}
    \item Discretize space into \( N \) grid points over the interval \([-a, a]\).
    \item Define the grid spacing:
    \[
    dx = \frac{2a}{N-1}.
    \]
\end{itemize}

\subsubsection*{Kinetic Energy Term (Matrix Representation)}
The second derivative in the kinetic energy term is represented using a finite difference approximation:
\[
\frac{d^2}{dx^2}\psi(x) \approx \frac{\psi(x+dx) - 2\psi(x) + \psi(x-dx)}{dx^2}.
\]
This leads to the tridiagonal matrix for the kinetic energy operator:
\[
T = -\frac{1}{2dx^2}
\begin{bmatrix}
-2 & 1 & 0 & \cdots & 0 \\
1 & -2 & 1 & \cdots & 0 \\
0 & 1 & -2 & \cdots & 0 \\
\vdots & \vdots & \vdots & \ddots & 1 \\
0 & 0 & 0 & 1 & -2
\end{bmatrix}.
\]

\subsubsection*{Potential Energy Term (Diagonal Matrix)}
The potential energy operator \( V(x) \) is diagonal, given by:
\[
V = \text{diag}\left(\frac{1}{2}x_1^2, \frac{1}{2}x_2^2, \ldots, \frac{1}{2}x_N^2\right),
\]
where \( x_i \) are the grid points.

\subsubsection*{Hamiltonian Matrix}
The full Hamiltonian matrix is the sum of the kinetic and potential energy matrices:
\[
H = T + V.
\]

\subsection*{Solving the Eigenvalue Problem}
The Schrödinger equation in matrix form is:
\[
H \vec{\psi} = E \vec{\psi},
\]
where:
\begin{itemize}
    \item \( \vec{\psi} \) is the eigenvector (discretized wavefunction).
    \item \( E \) is the eigenvalue (energy).
\end{itemize}
Numerical techniques (e.g., diagonalization) yield the eigenvalues and eigenvectors. The smallest eigenvalue corresponds to the ground-state energy \( E_0 \), and the associated eigenvector approximates \( \psi_0(x) \).

\subsection*{Validation Against Exact Solutions}
The exact solutions for the HO are:
\[
E_0 = \frac{1}{2}, \quad \psi_0(x) = Ce^{-x^2/2}.
\]
After solving numerically, compare the results to validate the accuracy of the numerical approach.

\section{Analytical Verification of the Ground State}
We now verify analytically that \( \psi(x) = Ce^{-ax^2} \) is a solution to the Schrödinger equation for some \( a > 0 \).

\subsection*{Step 1: Compute Derivatives of \( \psi(x) \)}
The trial wavefunction is:
\[
\psi(x) = Ce^{-ax^2}.
\]
The first derivative is:
\[
\psi'(x) = -2aCx e^{-ax^2}.
\]
The second derivative is:
\[
\psi''(x) = C\left(-2ae^{-ax^2} + 4a^2x^2e^{-ax^2}\right).
\]

\subsection*{Step 2: Substitute into Schrödinger Equation}
Substitute \( \psi(x) \), \( \psi'(x) \), and \( \psi''(x) \) into:
\[
-\frac{1}{2}\psi''(x) + \frac{1}{2}x^2\psi(x) = E\psi(x).
\]

The kinetic energy term is:
\[
-\frac{1}{2}\psi''(x) = -\frac{1}{2}C\left(-2ae^{-ax^2} + 4a^2x^2e^{-ax^2}\right) = 
\]
\[
Ca e^{-ax^2} - 2a^2Cx^2e^{-ax^2}.
\]

The potential energy term is:
\[
\frac{1}{2}x^2\psi(x) = \frac{1}{2}x^2Ce^{-ax^2}.
\]

Adding these terms together:
\[
-\frac{1}{2}\psi''(x) + \frac{1}{2}x^2\psi(x) = 
\]
\[
C\left(a - 2a^2x^2 + \frac{x^2}{2}\right)e^{-ax^2}.
\]

For \( \psi(x) \) to satisfy the Schrödinger equation, the coefficient of \( x^2 \) must vanish:
\[
-2a^2 + \frac{1}{2} = 0 \implies a = 1.
\]
The energy \( E \) is the constant term:
\[
E = a = \frac{1}{2}.
\]

\subsection*{Step 3: Verify Heisenberg Uncertainty Relation (HUR)}
The HUR states:
\[
\Delta x \Delta p \geq \frac{1}{2}.
\]
For the ground state \( \psi(x) \), calculate \( \Delta x \) and \( \Delta p \) using:
\[
\langle x^2 \rangle = \int_{-\infty}^\infty x^2|\psi(x)|^2 dx, \quad \langle x \rangle = 0,
\]
\[
\Delta x = \sqrt{\langle x^2 \rangle - \langle x \rangle^2}.
\]

Normalization gives \( C = \left(\frac{a}{\pi}\right)^{1/4} \). Using integrals for Gaussian functions, verify:
\[
\Delta x \Delta p = \frac{1}{2}.
\]

\section{Analytical Solution of a Trial Wavefunction}

We begin by analyzing the trial wavefunction:
\[
\psi(x) = Cx e^{-x^2 / 2}.
\]
This wavefunction is proposed as a solution to the Schrödinger equation for the harmonic oscillator:
\[
-\frac{1}{2} \psi''(x) + \frac{1}{2} x^2 \psi(x) = E \psi(x),
\]
where \( \psi(x) \) represents the state, and \( E \) is the energy. We aim to:
\begin{itemize}
    \item Verify that \( \psi(x) \) satisfies the equation.
    \item Determine the corresponding energy \( E \).
\end{itemize}

\subsection*{Step 1: Compute Derivatives of \( \psi(x) \)}
The first derivative of \( \psi(x) \) is:
\[
\psi'(x) = \frac{d}{dx} \left(Cx e^{-x^2 / 2}\right).
\]
Using the product rule:
\[
\psi'(x) = C \left(e^{-x^2 / 2} + x \cdot \frac{d}{dx} e^{-x^2 / 2}\right).
\]
The derivative of the exponential term is:
\[
\frac{d}{dx} e^{-x^2 / 2} = -x e^{-x^2 / 2}.
\]
Thus:
\[
\psi'(x) = C \left(e^{-x^2 / 2} - x^2 e^{-x^2 / 2}\right) = C e^{-x^2 / 2} (1 - x^2).
\]

The second derivative of \( \psi(x) \) is:
\[
\psi''(x) = \frac{d}{dx} \left(C e^{-x^2 / 2} (1 - x^2)\right).
\]
Using the product rule again:
\[
\psi''(x) = 
\]
\[
C \left(\frac{d}{dx} e^{-x^2 / 2} (1 - x^2) + e^{-x^2 / 2} \frac{d}{dx} (1 - x^2)\right).
\]
Substitute the derivatives:
\[
\psi''(x) = C \left(-x e^{-x^2 / 2} (1 - x^2) + e^{-x^2 / 2} (-2x)\right).
\]
Simplify:
\[
\psi''(x) = -C e^{-x^2 / 2} \left(x (1 - x^2) + 2x\right) = 
\]
\[
-C e^{-x^2 / 2} x (3 - x^2).
\]

\subsection*{Step 2: Substitute into Schrödinger Equation}
Substitute \( \psi(x) \), \( \psi'(x) \), and \( \psi''(x) \) into the equation:
\[
-\frac{1}{2} \psi''(x) + \frac{1}{2} x^2 \psi(x) = E \psi(x).
\]
First, calculate the kinetic energy term:
\[
-\frac{1}{2} \psi''(x) = \frac{1}{2} C e^{-x^2 / 2} x (3 - x^2).
\]
Next, calculate the potential energy term:
\[
\frac{1}{2} x^2 \psi(x) = \frac{1}{2} x^2 C e^{-x^2 / 2}.
\]

Combine the terms:
\[
\frac{1}{2} C e^{-x^2 / 2} x (3 - x^2) + \frac{1}{2} x^2 C e^{-x^2 / 2} = E C x e^{-x^2 / 2}.
\]
Factor out \( C e^{-x^2 / 2} x \):
\[
\frac{1}{2} x (3 - x^2) + \frac{1}{2} x^3 = E x.
\]
Simplify:
\[
\frac{1}{2} x (3 - x^2 + x^2) = E x.
\]
\[
\frac{1}{2} x (3) = E x.
\]
Divide through by \( x \) (valid for \( x \neq 0 \)):
\[
E = \frac{3}{2}.
\]

\subsection*{Result}
The energy associated with this trial wavefunction is:
\[
E = \frac{3}{2}.
\]

\section{Ladder Operators for the Harmonic Oscillator}

The harmonic oscillator Hamiltonian is:
\[
H = -\frac{\hbar^2}{2m} \frac{d^2}{dx^2} + \frac{1}{2} m \omega^2 x^2.
\]
We define the ladder (creation and annihilation) operators:
\[
a_\pm = \frac{1}{\sqrt{2\hbar m \omega}} \left(\mp i \hbar \frac{d}{dx} + m \omega x\right).
\]

\subsection*{Commutation Relations}
The operators satisfy the commutation relations:
\[
[a_-, a_+] = 1, \quad [H, a_\pm] = \pm \hbar \omega a_\pm.
\]

\subsection*{Ground State}
The ground state \( \psi_0(x) \) satisfies:
\[
a_- \psi_0(x) = 0.
\]
This implies:
\[
\left(-i \hbar \frac{d}{dx} + m \omega x\right) \psi_0(x) = 0.
\]
Solve the differential equation:
\[
\frac{d \psi_0}{dx} = -\frac{m \omega}{\hbar} x \psi_0.
\]
Integrate:
\[
\ln \psi_0 = -\frac{m \omega}{2\hbar} x^2 + \text{constant}.
\]
Thus:
\[
\psi_0(x) = A e^{-m \omega x^2 / 2\hbar}.
\]

The energy of the ground state is:
\[
E_0 = \frac{1}{2} \hbar \omega.
\]

\subsection*{Excited States}
The excited states are generated by applying the raising operator \( a_+ \):
\[
\psi_n(x) = A_n (a_+)^n \psi_0(x).
\]
The energy levels are:
\[
E_n = \hbar \omega \left(n + \frac{1}{2}\right), \quad n = 0, 1, 2, \dots
\]


\section{Variational Principle}

The variational principle states that for any trial wavefunction \( \psi(x) \) that is normalized (\( \langle \psi | \psi \rangle = 1 \)), the energy expectation value is always greater than or equal to the ground-state energy:
\[
E_{\text{trial}} = \frac{\langle \psi | \hat{H} | \psi \rangle}{\langle \psi | \psi \rangle} \geq E_1,
\]
where \( E_1 \) is the true ground-state energy. Equality holds only when \( \psi(x) \) is the exact ground-state wavefunction.

\section{Energy of a Trial Wavefunction in a Box}

We analyze a particle in a box with potential:
\[
U(x) = 
\begin{cases}
0, & |x| \leq \frac{1}{2}, \\
+\infty, & |x| > \frac{1}{2}.
\end{cases}
\]
The trial wavefunction is:
\[
\psi(x) = C\left(x^2 - \frac{1}{4}\right),
\]
where \( C \) is a normalization constant. 

\subsection*{1. Normalization of the Trial Wavefunction}
To normalize the wavefunction, we require:
\[
\langle \psi | \psi \rangle = \int_{-\frac{1}{2}}^{\frac{1}{2}} |\psi(x)|^2 dx = 1.
\]
Substituting \( \psi(x) \):
\[
\int_{-\frac{1}{2}}^{\frac{1}{2}} C^2 \left(x^2 - \frac{1}{4}\right)^2 dx = 1.
\]
Expanding \( \left(x^2 - \frac{1}{4}\right)^2 \):
\[
\left(x^2 - \frac{1}{4}\right)^2 = x^4 - \frac{1}{2}x^2 + \frac{1}{16}.
\]
The normalization condition becomes:
\[
C^2 \int_{-\frac{1}{2}}^{\frac{1}{2}} \left(x^4 - \frac{1}{2}x^2 + \frac{1}{16}\right) dx = 1.
\]
Using symmetry:
\[
\int_{-\frac{1}{2}}^{\frac{1}{2}} x^4 dx = 2 \int_{0}^{\frac{1}{2}} x^4 dx = \frac{1}{80}, 
\]
\[
\quad 
\int_{-\frac{1}{2}}^{\frac{1}{2}} x^2 dx = 2 \int_{0}^{\frac{1}{2}} x^2 dx = \frac{1}{12}.
\]
For the constant term:
\[
\int_{-\frac{1}{2}}^{\frac{1}{2}} \frac{1}{16} dx = \frac{1}{8}.
\]
Substituting:
\[
C^2 \left(\frac{1}{80} - \frac{1}{24} + \frac{1}{8}\right) = 1.
\]
Simplify:
\[
C^2 \left(\frac{3}{240} - \frac{10}{240} + \frac{30}{240}\right) = 1 \implies C^2 \cdot \frac{8}{240} = 1.
\]
Solve for \( C \):
\[
C^2 = 30 \implies C = \sqrt{30}.
\]

\subsection*{2. Expectation Value of the Hamiltonian}
The Hamiltonian for a particle in this box is purely kinetic:
\[
\hat{H} = -\frac{\hbar^2}{2m} \frac{d^2}{dx^2}.
\]
The expectation value of the energy is:
\[
\langle \psi | \hat{H} | \psi \rangle = -\frac{\hbar^2}{2m} \int_{-\frac{1}{2}}^{\frac{1}{2}} \psi(x) \frac{d^2}{dx^2} \psi(x) dx.
\]

First, compute the second derivative of \( \psi(x) \):
\[
\psi(x) = \sqrt{30} \left(x^2 - \frac{1}{4}\right), \quad \frac{d\psi}{dx} = 
\]
\[
\sqrt{30} \cdot 2x, \quad \frac{d^2 \psi}{dx^2} = \sqrt{30} \cdot 2.
\]
Substitute into the Hamiltonian:
\[
\langle \psi | \hat{H} | \psi \rangle = -\frac{\hbar^2}{2m} \int_{-\frac{1}{2}}^{\frac{1}{2}} \sqrt{30} \left(x^2 - \frac{1}{4}\right) \sqrt{30} \cdot 2 dx.
\]
Simplify:
\[
\langle \psi | \hat{H} | \psi \rangle = -\frac{\hbar^2}{2m} \cdot 30 \int_{-\frac{1}{2}}^{\frac{1}{2}} 2\left(x^2 - \frac{1}{4}\right) dx.
\]
Using previous integrals:
\[
\int_{-\frac{1}{2}}^{\frac{1}{2}} x^2 dx = \frac{1}{12}, \quad \int_{-\frac{1}{2}}^{\frac{1}{2}} dx = 1.
\]
Substitute:
\[
\langle \psi | \hat{H} | \psi \rangle = -\frac{\hbar^2}{2m} \cdot 30 \cdot 2 \left(\frac{1}{12} - \frac{1}{4}\right).
\]
Simplify:
\[
\langle \psi | \hat{H} | \psi \rangle = -\frac{\hbar^2}{2m} \cdot 30 \cdot 2 \cdot \left(-\frac{1}{6}\right).
\]
\[
\langle \psi | \hat{H} | \psi \rangle = \frac{\hbar^2}{2m} \cdot 10 = \frac{5\hbar^2}{m}.
\]

\subsection*{3. True Ground-State Energy}
The true ground-state energy for a particle in a box is:
\[
E_{\text{ground}} = \frac{\pi^2 \hbar^2}{2mL^2}.
\]
For \( L = 1 \) (from \(-\frac{1}{2}\) to \(\frac{1}{2}\)):
\[
E_{\text{ground}} = \frac{\pi^2 \hbar^2}{2m}.
\]

\subsection*{Comparison}
The trial energy is:
\[
E_{\text{trial}} = \frac{5\hbar^2}{m}, \quad E_{\text{ground}} = \frac{\pi^2 \hbar^2}{2m} \approx \frac{3.15 \hbar^2}{m}.
\]
Thus:
\[
E_{\text{trial}} > E_{\text{ground}}.
\]
This is consistent with the variational principle.

\end{multicols}
\end{document}
