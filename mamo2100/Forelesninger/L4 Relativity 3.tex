\documentclass[a4paper,12pt]{article}
\usepackage{multicol}
\usepackage{geometry}
\usepackage{amsmath}
\usepackage{amssymb}
\usepackage{titlesec}
\geometry{a4paper, left=0.5in, right=0.5in, top=0.5in, bottom=0.5in}
\setlength{\parindent}{0pt}
\setlength{\columnsep}{1cm}

% Adjust section title font size and spacing
\titleformat{\section}
  {\normalfont\normalsize\bfseries} % Set section title font to small and bold
  {}{0pt}{}
\titlespacing*{\section}{0pt}{0.3cm}{0.2cm} % Adjust space before and after section

\begin{document}
\title{L4 - Relativity - Momentum - Doppler-effect}
\author{}
\date{}
\maketitle
\begin{multicols}{2}

\section*{Relativistic Force}
\textbf{Newton's Second Law: Classical Mechanics}  
In classical mechanics, Newton's second law is expressed as:
\[
F = \frac{d\mathbf{p}}{dt} = \frac{d}{dt}(m\mathbf{v}) = m\frac{d\mathbf{v}}{dt} = m\mathbf{a}.
\]
Here:
\begin{itemize}
    \item \( F \): Force
    \item \( \mathbf{p} = m\mathbf{v} \): Momentum (mass times velocity)
    \item \( \mathbf{a} = \frac{d\mathbf{v}}{dt} \): Acceleration
\end{itemize}

\textbf{Newton's Second Law: Relativistic Mechanics}  
For relativistic mechanics, the momentum is defined as:
\[
\mathbf{p}_{\text{rel}} = \frac{m\mathbf{v}}{\sqrt{1 - \frac{v^2}{c^2}}}.
\]
Thus, the force becomes:
\[
F = \frac{d}{dt}\left( \frac{m\mathbf{v}}{\sqrt{1 - \frac{v^2}{c^2}}} \right).
\]

\section*{Derivation: Relativistic Force Along One Axis}
Given that the velocity and force act along the \(x\)-axis, let \( \mathbf{v} = v \) and \( \mathbf{a} = \frac{dv}{dt} \). We derive:
\[
F = m\frac{d}{dt}\left(v \sqrt{1 - \frac{v^2}{c^2}}^{-1}\right).
\]
Using the product rule, expand:
\[
\frac{d}{dt}\left[v (1 - \frac{v^2}{c^2})^{-1/2}\right] = 
\]
\[
v\frac{d}{dt}(1 - \frac{v^2}{c^2})^{-1/2} + (1 - \frac{v^2}{c^2})^{-1/2}\frac{dv}{dt}.
\]
Now compute each term:
\begin{enumerate}
    \item For the derivative of \((1 - \frac{v^2}{c^2})^{-1/2}\), use the chain rule:
    \[
    \frac{d}{dt}(1 - \frac{v^2}{c^2})^{-1/2} = -\frac{1}{2}(1 - \frac{v^2}{c^2})^{-3/2}\cdot\frac{d}{dt}(-\frac{v^2}{c^2}).
    \]
    Since \(\frac{d}{dt}(-\frac{v^2}{c^2}) = -\frac{2v}{c^2}\frac{dv}{dt}\), this becomes:
    \[
    \frac{d}{dt}(1 - \frac{v^2}{c^2})^{-1/2} = \frac{v}{c^2}(1 - \frac{v^2}{c^2})^{-3/2}a.
    \]
    \item Substitute back into \(F\):
    \[
    F = 
    \]
    \[
    m\left[a(1 - \frac{v^2}{c^2})^{-3/2} + \frac{v^2}{c^2}a(1 - \frac{v^2}{c^2})^{-3/2}\right]
    \]
    Combine terms:
    \[
    F = \frac{ma}{(1 - \frac{v^2}{c^2})^{3/2}}.
    \]
\end{enumerate}

\section*{Relativistic Momentum}
To ensure consistency with relativity, momentum is redefined as:
\[
p = \frac{m v}{\sqrt{1 - \frac{v^2}{c^2}}},
\]
where:
\begin{itemize}
    \item \(m\): Rest mass
    \item \(v\): Speed
    \item \(c\): Speed of light.
\end{itemize}

For small velocities (\(v \ll c\)), we approximate:
\[
\frac{1}{\sqrt{1 - \frac{v^2}{c^2}}} \approx 1 + \frac{1}{2}\frac{v^2}{c^2}.
\]
Thus:
\[
p_{\text{rel}} \approx m v + \frac{1}{2}m\frac{v^3}{c^2}.
\]
The relativistic momentum approaches infinity as \(v \to c\), since \(1 - \frac{v^2}{c^2} \to 0\).

\section*{Speed of an Electron with Enhanced Momentum}
If the relativistic momentum is \(x\) times the classical momentum:
\[
\frac{mv}{\sqrt{1 - \frac{v^2}{c^2}}} = x(mv).
\]
Simplify:
\[
\frac{1}{\sqrt{1 - \frac{v^2}{c^2}}} = x.
\]
Square both sides:
\[
1 = x^2(1 - \frac{v^2}{c^2}).
\]
Solve for \(v^2\):
\[
\frac{v^2}{c^2} = 1 - \frac{1}{x^2}.
\]
Take the square root:
\[
v = c\sqrt{1 - \frac{1}{x^2}}.
\]

\section*{Relativistic Doppler Effect}
\textbf{Frequency of Waves Received by an Observer:}  
Consider a source emitting light waves at frequency \(f_0\) in its rest frame, moving with speed \(u\) toward a stationary observer. The observed frequency \(f\) is given by:
\[
f = \sqrt{\frac{c+u}{c-u}}f_0.
\]

\textbf{Derivation:}
Start from the relation \(f = \frac{c}{\lambda}\), where \(\lambda = cT - uT = (c-u)T\). Thus:
\[
f = \frac{c}{(c-u)T}.
\]
The relativistic time dilation gives \(T = \gamma T_0\), where \(\gamma = \frac{1}{\sqrt{1 - \frac{u^2}{c^2}}}\). Substituting:
\[
f = \frac{c}{c-u}\cdot\frac{1}{T_0}.
\]
Using \(f_0 = \frac{1}{T_0}\), we arrive at:
\[
f = \sqrt{\frac{c+u}{c-u}}f_0.
\]

\textbf{Remarks:}
\begin{itemize}
    \item When the source moves \textit{toward} the observer (\(u > 0\)), \(f > f_0\) (blue shift).
    \item When the source moves \textit{away} (\(u < 0\)), \(f < f_0\) (red shift).
\end{itemize}
\section*{Finding the Speed of a Particle with Momentum Greater than Classical Momentum}

\subsection*{Problem}
An electron has a momentum that is 90\% larger than its classical momentum. Find the speed of the electron. How would the result change for a proton?

\subsection*{Solution}
The relativistic momentum is given by:
\[
p_{\text{rel}} = x p_{\text{cl}}
\]
where \(x = 1.9\) (since the momentum is 90\% larger). The classical momentum \(p_{\text{cl}}\) is:
\[
p_{\text{cl}} = mv
\]
The relativistic momentum \(p_{\text{rel}}\) is expressed as:
\[
p_{\text{rel}} = \frac{mv}{\sqrt{1 - v^2/c^2}}
\]
Equating \(p_{\text{rel}} = x p_{\text{cl}}\), we have:
\[
\frac{mv}{\sqrt{1 - v^2/c^2}} = x (mv)
\]
The mass \(m\) cancels out, leaving:
\[
\frac{1}{\sqrt{1 - v^2/c^2}} = x
\]
Squaring both sides:
\[
\frac{1}{1 - v^2/c^2} = x^2
\]
Rearranging for \(v^2\):
\[
1 - v^2/c^2 = \frac{1}{x^2}
\]
\[
v^2/c^2 = 1 - \frac{1}{x^2}
\]
\[
v = c \sqrt{1 - \frac{1}{x^2}}
\]
Substitute \(x = 1.9\):
\[
v = c \sqrt{1 - \frac{1}{1.9^2}}
\]
\[
v = c \sqrt{1 - 0.277}
\]
\[
v \approx c \sqrt{0.723}
\]
\[
v \approx 0.85c
\]

\subsection*{Result for a Proton}
The calculation remains the same since the mass cancels out in the derivation. Therefore, the speed of the proton would also be approximately \(0.85c\).

\section*{Relating Force to Acceleration in Relativistic Mechanics}

\subsection*{Problem}
Show that \(F = \gamma^3 ma\), where \(\gamma = \frac{1}{\sqrt{1 - v^2/c^2}}\), using \(F = \frac{d}{dt}p\), \(p = \frac{mv}{\sqrt{1 - v^2/c^2}}\), and \(a = \frac{dv}{dt}\). Compute \(\lim_{v \to c} a\).

\subsection*{Solution}
The relativistic force is:
\[
F = \frac{d}{dt}p
\]
The momentum \(p\) is:
\[
p = \frac{mv}{\sqrt{1 - v^2/c^2}}
\]
Taking the time derivative:
\[
F = \frac{d}{dt} \left( \frac{mv}{\sqrt{1 - v^2/c^2}} \right)
\]
Using the product rule:
\[
F = m \frac{d}{dt}\left( \frac{v}{\sqrt{1 - v^2/c^2}} \right)
\]
The derivative of \(v/\sqrt{1 - v^2/c^2}\) involves the chain rule:
\[
\frac{d}{dt} \left( \frac{v}{\sqrt{1 - v^2/c^2}} \right) =
\]
\[
\frac{\sqrt{1 - v^2/c^2} \frac{dv}{dt} + v \frac{d}{dt} \sqrt{1 - v^2/c^2}}{(1 - v^2/c^2)}
\]
The derivative of \(\sqrt{1 - v^2/c^2}\) is:
\[
\frac{d}{dt} \sqrt{1 - v^2/c^2} = -\frac{1}{2} \frac{1}{\sqrt{1 - v^2/c^2}} \cdot \frac{d}{dt}(v^2/c^2)
\]
Substituting, simplifying, and factoring terms yields:
\[
F = \gamma^3 ma
\]

As \(v \to c\), \(\gamma \to \infty\), and the acceleration \(a \to 0\) due to the increase in inertial mass.

\section*{Relativistic Doppler Effect and Velocity}

\subsection*{Problem}
Using the relativistic Doppler effect, show that:
\[
\frac{u}{c} = \frac{\lambda_0^2 - \lambda^2}{\lambda_0^2 + \lambda^2}
\]

\subsection*{Solution}
The relativistic Doppler shift for a source moving at speed \(u\) is:
\[
\frac{\lambda}{\lambda_0} = \sqrt{\frac{1 - u/c}{1 + u/c}}
\]
Squaring both sides:
\[
\left( \frac{\lambda}{\lambda_0} \right)^2 = \frac{1 - u/c}{1 + u/c}
\]
Cross-multiplying:
\[
\lambda^2 (1 + u/c) = \lambda_0^2 (1 - u/c)
\]
Expanding terms:
\[
\lambda^2 + \frac{u}{c} \lambda^2 = \lambda_0^2 - \frac{u}{c} \lambda_0^2
\]
Rearranging to isolate \(\frac{u}{c}\):
\[
\frac{u}{c} (\lambda^2 + \lambda_0^2) = \lambda_0^2 - \lambda^2
\]
\[
\frac{u}{c} = \frac{\lambda_0^2 - \lambda^2}{\lambda_0^2 + \lambda^2}
\]
\end{multicols}
\end{document}
