\documentclass[a4paper,12pt]{article}
\usepackage{multicol}
\usepackage{geometry}
\usepackage{amsmath}
\usepackage{titlesec} % To customize section title size and spacing
\geometry{a4paper, left=0.5in, right=0.5in, top=0.5in, bottom=0.5in}
\setlength{\parindent}{0pt} % Remove paragraph indentation
\setlength{\columnsep}{1cm} % Set spacing between columns

% Adjust section title font size and spacing
\titleformat{\section}
  {\normalfont\normalsize\bfseries} % Set section title font to small and bold
  {}{0pt}{}
\titlespacing*{\section}{0pt}{0.3cm}{0.2cm} % Adjust space before and after section

\begin{document}

\title{L3 - Lorentz Velocity Transformation}
\author{}
\date{}
\maketitle

\begin{multicols}{2}

\section*{Example: Spaceship and Probe}
A spaceship moving away from Earth at $0.9c$ fires a probe at $0.7c$ relative to the spaceship. What is the probe's velocity relative to Earth?

\textbf{Solution:}

The reference frames are defined as follows:
\begin{itemize}
  \item $S$: Earth as the reference frame.
  \item $S'$: Spaceship as the reference frame.
\end{itemize}

Given:

\begin{itemize}
  \item $u = 0.9c$ (velocity of the spaceship relative to Earth)
  \item $v'_x = 0.7c$ (velocity of the probe relative to the spaceship)
\end{itemize}


We use the Lorentz velocity transformation formula:
\begin{equation}
  v_x = \frac{v'_x + u}{1 + \frac{u v'_x}{c^2}}
\end{equation}

Substituting the values:
\begin{align*}
  v_x &= \frac{0.7c + 0.9c}{1 + \frac{0.9c \cdot 0.7c}{c^2}} \\
      &= \frac{1.6c}{1 + 0.63} \\
      &= \frac{1.6c}{1.63} \\
      &= 0.98c
\end{align*}

\textbf{Comment:} Note that $v_x < c$, consistent with the relativistic principle that no object can exceed the speed of light.

\section*{Galilean Transformation Comparison}
If we incorrectly use the Galilean velocity transformation, we get:
\begin{equation}
  v_x = v'_x + u = 0.7c + 0.9c = 1.6c
\end{equation}
This result contradicts the relativistic principle, as it suggests $v_x > c$.

\section*{Deriving the Lorentz Velocity Transformation}
Consider the Lorentz velocity transformation:
\begin{equation}
  v'_x = \frac{v_x - u}{1 - \frac{u v_x}{c^2}}
\end{equation}
Solving for $v_x$ algebraically yields:
\begin{equation}
  v_x = \frac{v'_x + u}{1 + \frac{u v'_x}{c^2}}
\end{equation}
This formula is used in the previous example to determine the velocity of the probe relative to Earth.

\textbf{Interpretation:} The result is consistent with swapping $v_x \leftrightarrow v'_x$ and reversing the sign of $u$, as expected due to the symmetry of Lorentz transformations.

\section*{Summary: Lorentz Velocity Transformation}
The Lorentz velocity transformation is given by:
\begin{equation}
  v'_x = \frac{v_x - u}{1 - \frac{u v_x}{c^2}}
\end{equation}

\textbf{Limiting Cases:}
\begin{enumerate}
  \item \textbf{Low Velocities ($u \ll c$, $v_x \ll c$):}
  \begin{align*}
    1 - \frac{u v_x}{c^2} &\approx 1 \\
    v'_x &\approx v_x - u \quad \text{(Galilean approximation)}
  \end{align*}

  \item \textbf{$v_x = c$:}
  \begin{align*}
    v'_x &= \frac{c - u}{1 - \frac{u c}{c^2}} = \frac{c - u}{\frac{c - u}{c}} = c
  \end{align*}
  In this case, $v'_x = v_x = c$, which is consistent with Einstein's postulate that the speed of light is the same in all inertial frames of reference.
\end{enumerate}

\section*{Derivation of the Lorentz Transformation}
To derive the Lorentz velocity transformation, we start by recalling the Galilean velocity transformation:
\begin{equation}
  x = x' + u t
\end{equation}
Differentiating with respect to time gives:
\begin{equation}
  v_x = v'_x + u
\end{equation}
This relation, however, is inconsistent with special relativity for high velocities.

The Lorentz transformation is given by:
\begin{align*}
  x' &= \gamma (x - ut) \\
  t' &= \gamma \left( t - \frac{ux}{c^2} \right)
\end{align*}
where $\gamma = \frac{1}{\sqrt{1 - \frac{u^2}{c^2}}}$.

Using the chain rule for differentiation:
\begin{align*}
  \frac{dx'}{dt'} &= \frac{\partial x'}{\partial x} \cdot \frac{dx}{dt} + \frac{\partial x'}{\partial t} \cdot \frac{dt}{dt} \\
                   &= \gamma \left( \frac{dx}{dt} - u \right)
\end{align*}
Dividing by $\frac{dt'}{dt}$ yields:
\begin{equation}
  v'_x = \frac{v_x - u}{1 - \frac{u v_x}{c^2}}
\end{equation}

\section*{Matrix Form of Lorentz Transformation}
We can also express the Lorentz transformation in matrix form. Let:
\begin{align*}
  y_1 &= a x_1 + b x_2 \\
  y_2 &= c x_1 + d x_2
\end{align*}
In matrix form:
\begin{equation}
  \begin{pmatrix} y_1 \\ y_2 \end{pmatrix} = \begin{pmatrix} a & b \\ c & d \end{pmatrix} \begin{pmatrix} x_1 \\ x_2 \end{pmatrix}
\end{equation}
For the Lorentz transformation:
\begin{equation}
  \begin{pmatrix} x' \\ t' \end{pmatrix} = \begin{pmatrix} \gamma & -\gamma u \\ -\frac{\gamma u}{c^2} & \gamma \end{pmatrix} \begin{pmatrix} x \\ t \end{pmatrix}
\end{equation}

\section*{Summary: Lorentz Transformation}
The Lorentz transformations for space and time are:
\begin{align*}
  x' &= \gamma (x - ut) \\
  t' &= \gamma \left( t - \frac{ux}{c^2} \right)
\end{align*}
These equations are the relativistic generalizations of the Galilean transformations:
\begin{align*}
  x &= x' + ut \\
  t &= t'
\end{align*}
Space and time have become intertwined, and we can no longer say that length and time have absolute meanings independent of the frame of reference.

\section*{Differential Formula Application}
Let $a, b$ be constants. By using the differential formula for a function of two variables $f(x, t) = ax + bt$:
\begin{equation}
  df = \frac{\partial f}{\partial x} dx + \frac{\partial f}{\partial t} dt,
\end{equation}
show that this leads to:
\begin{equation}
  df = a dx + b dt.
\end{equation}

\textbf{Solution:} Since
\begin{align*}
  \frac{\partial f}{\partial x} &= a, \quad \frac{\partial f}{\partial t} = b
\end{align*}
we get the desired result directly.

\section*{Relative Velocity Between Spaceships}
Consider two spaceships, A and B, moving in opposite directions towards each other. An observer on Earth measures the speed of A to be $0.75c$ and that of B to be $0.85c$. Find the velocity of B with respect to A.

\textbf{Hint:} Let $S'$ be the frame attached to spaceship A, so that $u = 0.75c$ is the relative motion to an observer on Earth ($S$ frame). Think carefully about the sign of $v_x$ being the velocity of spaceship B in the $S$ frame.

\textbf{Solution:} Using the hint, we have that $v_x = -0.85c$ (velocity relative to Earth). The velocity of B with respect to A can then be computed from:
\begin{equation}
  v'_x = \frac{v_x - u}{1 - \frac{u v_x}{c^2}} = \frac{-0.85 - 0.75}{1 - 0.75 \cdot (-0.85)} c = -0.9771c.
\end{equation}

\section*{Solving for $v_x$ in the Lorentz Expression}
Solve for $v_x$ in the expression:
\begin{equation}
  v'_x = \frac{v_x - u}{1 - \frac{u v_x}{c^2}}.
\end{equation}

\textbf{Solution:} We first multiply both sides by $1 - \frac{u v_x}{c^2}$ such that:
\begin{equation}
  v'_x \left( 1 - \frac{u v_x}{c^2} \right) = v_x - u.
\end{equation}
We now isolate $v_x$:
\begin{align*}
  v'_x + u &= v_x + \frac{u}{c^2} v'_x v_x = \left( 1 + \frac{u v'_x}{c^2} \right) v_x
\end{align*}
to get the answer:
\begin{equation}
  v_x = \frac{v'_x + u}{1 + \frac{u v'_x}{c^2}}.
\end{equation}

\end{multicols}
\end{document}
