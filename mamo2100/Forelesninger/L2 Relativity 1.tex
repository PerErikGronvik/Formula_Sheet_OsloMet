\documentclass[a4paper,12pt]{article}
\usepackage{amsmath}
\usepackage{geometry}
\usepackage{multicol}
\usepackage{titlesec}

\geometry{a4paper, left=0.5in, right=0.5in, top=0.5in, bottom=0.5in}
\setlength{\parindent}{0pt}
\setlength{\columnsep}{1cm}

\titleformat{\section}
  {\normalfont\normalsize\bfseries}
  {}{0pt}{}
\titlespacing*{\section}{0pt}{0.3cm}{0.2cm}

\begin{document}
\begin{center}
    \textbf{\large L2 Relativity - Length Contraction - Time Dilation - Lorentz Factor, Time Intervals}
\end{center}

\vspace{0.2cm}
\begin{multicols}{2}

\section*{Length Contraction}
Length contraction is given by the formula:
\[
l = \sqrt{1 - \frac{u^2}{c^2}} \cdot l_0,
\]
where \( l_0 \) is the rest length and \( u \) is the velocity of the moving object.

Alternatively, we can write:
\[
l = \frac{l_0}{\gamma}, \quad \gamma = \frac{1}{\sqrt{1 - \frac{u^2}{c^2}}},
\]
where \( \gamma \) is the **Lorentz factor**.

Length contraction only occurs for lengths measured in the direction **parallel** to the relative motion (\(u\)). Lengths that are measured **perpendicular** to the direction of motion are **not contracted**.

For small velocities (\(u \ll c\)):
\[
\frac{1}{\gamma} = \sqrt{1 - \frac{u^2}{c^2}} \approx 1 - \frac{1}{2}\frac{u^2}{c^2} + \cdots
\]
This implies that for non-relativistic speeds (\(u \ll c\)), the length contraction is negligible:
\[
l \approx l_0.
\]
For the relativistic case:
\[
l = \left( 1 - \frac{1}{2} \frac{u^2}{c^2} + \cdots \right) l_0 = l_0 - \frac{1}{2} \frac{u^2}{c^2} l_0 + \cdots
\]

\subsection*{Example}
A spaceship is measured to be \( 100 \, \text{m} \) long when at rest with respect to an observer. If the spaceship flies by the observer with speed \( 0.99c \), what length will the observer find for the spaceship?

\textbf{Solution:}

Using length contraction:
\[
\Delta l = l_0 \sqrt{1 - \frac{u^2}{c^2}},
\]
where \( l_0 = 100 \, \text{m} \) and \( u = 0.99c \):
\[
\frac{u^2}{c^2} = \left( 0.99c \right)^2 / c^2 = 0.99^2,
\]
\[
\Delta l = 100 \cdot \sqrt{1 - 0.99^2} \approx 14 \, \text{m}.
\]

\section*{Relativity of Length}
Length contraction occurs for lengths parallel to the direction of motion.

Consider a frame \( S \) and a moving frame \( S' \). Let \( l_0 \) be the rest length.

\[
d = l + u \Delta t_1, \quad d = c \Delta t_1 \quad \Rightarrow \quad \Delta t_1 = \frac{l}{c - u}
\]
\[
d = l - u \Delta t_2, \quad d = c \Delta t_2 \quad \Rightarrow \quad \Delta t_2 = \frac{l}{c + u}
\]

Total time:
\[
\Delta t = \Delta t_1 + \Delta t_2 = \frac{l}{c - u} + \frac{l}{c + u}
\]
\[
= \frac{l (c + u) + l (c - u)}{c^2 - u^2} = \frac{2lc}{c^2 - u^2}
\]
\[
= \frac{2l/c}{1 - \frac{u^2}{c^2}} = \frac{2l}{c(1 - \frac{u^2}{c^2})} \tag{L5}
\]

\section*{Time Dilation}
Two events occur at the same point in space:
- Observer at rest: \( \Delta t_0 \) (rest frame)
- Observer moving with constant speed \( u \): \( \Delta t \) (relative frame)

\[
\Delta t = \frac{\Delta t_0}{\sqrt{1 - \frac{u^2}{c^2}}}, \quad u < c
\]

\subsection*{Definition (Lorentz Factor)}
\[
\gamma = \frac{1}{\sqrt{1 - \frac{u^2}{c^2}}}, \quad \beta = \frac{u}{c}
\]
\[
\gamma = \frac{1}{\sqrt{1 - \beta^2}}
\]

For small velocities (\( u \ll c \)):
\[
\gamma = (1 - \beta^2)^{-1/2} \approx 1 + \frac{1}{2} (\beta^2) + \cdots
\]
Thus:
\[
\Delta t \approx \Delta t_0 \quad (\text{non-relativistic})
\]
\[
\Delta t = \left( 1 + \frac{1}{2} \frac{u^2}{c^2} + \cdots \right) \Delta t_0 = \Delta t_0 + \frac{1}{2} \frac{u^2}{c^2} \Delta t_0 + \cdots
\]

\subsection*{Example}
Person A is in a flying spaceship with a watch and notes that after \( 3600 \, \text{s} \), on earth for person B watching the spaceship, \( 3601 \, \text{s} \) has passed. At what speed is the spaceship flying?

\textbf{Solution:}
\[
\Delta t = \frac{\Delta t_0}{\sqrt{1 - \frac{u^2}{c^2}}}
\]
\[
\left( \Delta t \right)^2 = \frac{\left( \Delta t_0 \right)^2}{1 - \frac{u^2}{c^2}}
\]
Rearranging:
\[
\frac{u^2}{c^2} = 1 - \frac{(\Delta t_0)^2}{(\Delta t)^2}
\]
\[
u = c \sqrt{1 - \frac{(\Delta t_0)^2}{(\Delta t)^2}}
\]
Given \( \Delta t_0 = 3600 \, \text{s} \) and \( \Delta t = 3601 \, \text{s} \):
\[
u = 3 \times 10^8 \sqrt{1 - \frac{3600^2}{3601^2}} \approx 7 \times 10^6 \, \text{m/s}
\]

\section*{Limitations of Galilean Transformation and Necessity for Lorentz Transformation}
The **Galilean transformation** describes how coordinates change between two inertial frames:
\[
x = x' + ut, \quad y = y',
\]
where \(u\) is the constant relative speed.

This transformation is **not consistent with relativity**, since it implies that velocities are additive:
\[
v_x = v'_x + u,
\]
which can result in velocities greater than the speed of light (\(c\)), violating relativistic constraints.

To resolve this issue, we use the **Lorentz transformation**, which ensures that no object can travel faster than the speed of light and properly takes into account time dilation and length contraction.

\section*{Relativity of Time Intervals}
The time interval may be different in different frames of reference.

Consider two frames \( S \) and \( S' \):
\[
\Delta t_0 = \frac{2d}{c}
\]
Using Pythagoras:
\[
l^2 = d^2 + \left( \frac{u \Delta t}{2} \right)^2, \quad l = \sqrt{d^2 + \left( \frac{u \Delta t}{2} \right)^2}
\]
\[
\Delta t = \frac{2l}{c}
\]
Squaring both sides:
\[
(\Delta t)^2 = \frac{4}{c^2} d^2 + \frac{4}{c^2} \left( \frac{u \Delta t}{2} \right)^2 = \left( \frac{2d}{c} \right)^2 + \frac{u^2 (\Delta t)^2}{c^2}
\]
Rearranging gives:
\[
(\Delta t)^2 \left( 1 - \frac{u^2}{c^2} \right) = (\Delta t_0)^2
\]
Thus:
\[
\Delta t = \frac{\Delta t_0}{\sqrt{1 - \frac{u^2}{c^2}}}, \quad u < c
\]

\section*{Einstein's Postulates}
1. **Principle of Relativity**: The laws of physics are the same in every inertial frame of reference.
2. **Constancy of Speed of Light**: The speed of light in a vacuum is the same in all inertial frames of reference.

The speed of light \(c = 3 \times 10^8 \, \text{m/s}\). It is impossible for an inertial observer to travel at the speed of \(c\).

\section*{Relativity 1}
Let \( c \) denote the speed of light. Moreover, we sometimes use the abbreviations for:
- **Lorentz factor** (\( \gamma \)):
\[
\gamma = \frac{1}{\sqrt{1 - \frac{u^2}{c^2}}}, \quad \text{where } u < c,
\]
- **Velocity ratio** (\( \beta \)):
\[
\beta = \frac{u}{c}, \tag{1}
\]
which simplifies expressions in relativistic physics.

\section*{Calculating the Speed of a Spaceship Using Time Dilation}
Person A is in a flying spaceship with a watch and notes that after \( \Delta t_0 \) seconds, on earth for person B watching the spaceship, a time interval of \( \Delta t > \Delta t_0 \) seconds has elapsed. At what speed is the spaceship flying? (Answer in terms of \( \Delta t_0 \) and \( \Delta t \)).

\textbf{Solution:} We denote the time interval where the watch is at rest as \( \Delta t_0 \) (proper time). 

The time dilation formula relates the proper time (\( \Delta t_0 \)) to the time observed in another frame (\( \Delta t \)):
\[
\Delta t = \frac{\Delta t_0}{\sqrt{1 - \frac{u^2}{c^2}}}. \tag{L1}
\]

We want to solve for \( u \) in terms of \( \Delta t \) and \( \Delta t_0 \).

First, isolate the term involving \( u \) by squaring both sides:
\[
(\Delta t)^2 = \frac{(\Delta t_0)^2}{1 - \frac{u^2}{c^2}}.
\]

Rearrange to solve for \( u^2 \):
\[
1 - \frac{u^2}{c^2} = \frac{(\Delta t_0)^2}{(\Delta t)^2},
\]
\[
\frac{u^2}{c^2} = 1 - \frac{(\Delta t_0)^2}{(\Delta t)^2}. \tag{L2}
\]

Take the square root to solve for \( u \):
\[
u = c \sqrt{1 - \frac{(\Delta t_0)^2}{(\Delta t)^2}}. \tag{L3}
\]

This formula provides the speed of the spaceship in terms of the observed and proper times.

\section*{Finding the Speed for a Specific Time Dilation Ratio}
How quickly does the spaceship in the above scenario need to travel (relative to earth) such that time on the spaceship goes half as quickly as time on earth?

\textbf{Solution:} From the time dilation formula:
\[
\Delta t = \frac{\Delta t_0}{\sqrt{1 - \frac{u^2}{c^2}}}. \tag{L4}
\]

We want the time on the spaceship to be half of the time on earth:
\[
\Delta t = 2 \Delta t_0.
\]

Plugging into the time dilation formula:
\[
2 \Delta t_0 = \frac{\Delta t_0}{\sqrt{1 - \frac{u^2}{c^2}}}.
\]

Divide both sides by \( \Delta t_0 \):
\[
2 = \frac{1}{\sqrt{1 - \frac{u^2}{c^2}}}.
\]

Square both sides to solve for \( u^2 \):
\[
4 = \frac{1}{1 - \frac{u^2}{c^2}},
\]
\[
1 - \frac{u^2}{c^2} = \frac{1}{4}.
\]

Rearrange to solve for \( \frac{u^2}{c^2} \):
\[
\frac{u^2}{c^2} = 1 - \frac{1}{4} = \frac{3}{4}. \tag{L5}
\]

Take the square root to find \( u \):
\[
u = \sqrt{\frac{3}{4}} c = \left( \frac{\sqrt{3}}{2} \right) c \approx 0.866c.
\]

Thus, the speed of the spaceship is \( u = (\sqrt{3}/2) c \, \text{m/s} \).

\end{multicols}
\end{document}
