\documentclass[a4paper,12pt]{article}
\usepackage{multicol}
\usepackage{geometry}
\usepackage{amsmath}
\usepackage{amssymb}
\usepackage{titlesec} % To customize section title size and spacing
\geometry{a4paper, left=0.5in, right=0.5in, top=0.5in, bottom=0.5in}
\setlength{\parindent}{0pt} % Remove paragraph indentation
\setlength{\columnsep}{1cm} % Set spacing between columns

% Adjust section title font size and spacing
\titleformat{\section}
  {\normalfont\normalsize\bfseries} % Set section title font to small and bold
  {}{0pt}{}
\titlespacing*{\section}{0pt}{0.3cm}{0.2cm} % Adjust space before and after section

\begin{document}
\title{L7 - Quantum Mechanics: Particle in a Box}
\author{}
\date{}
\maketitle
\begin{multicols}{2}

\section*{The Schrödinger Equation}
The time-independent Schrödinger equation (SE) for a 1D quantum system is:
\[
-\frac{\hbar^2}{2m} \frac{d^2\psi(x)}{dx^2} + U(x)\psi(x) = E\psi(x),
\]
where:
\begin{itemize}
    \item $\psi(x)$: Wavefunction of the particle.
    \item $U(x)$: Potential energy function.
    \item $E$: Energy eigenvalue (observable energy).
    \item $\hbar$: Reduced Planck's constant.
\end{itemize}
The wavefunction $\psi(x)$ must satisfy normalization:
\[
\int_{-\infty}^\infty |\psi(x)|^2 dx = 1,
\]
ensuring the total probability of finding the particle is 1.

\section*{Particle in a Box (Infinite Square Well)}
We consider a particle confined in a 1D box of length $L$ with infinite potential walls:
\[
U(x) = 
\begin{cases} 
0, & 0 \leq x \leq L, \\
+\infty, & \text{otherwise}.
\end{cases}
\]
Boundary conditions: $\psi(0) = \psi(L) = 0$ (wavefunction vanishes at the walls).

\section*{Solving the Schrödinger Equation}
The SE simplifies in the box region ($U(x) = 0$):
\[
\frac{d^2\psi(x)}{dx^2} + k^2\psi(x) = 0, \quad k^2 = \frac{2mE}{\hbar^2}.
\]
General solution:
\[
\psi(x) = C_1 \cos(kx) + C_2 \sin(kx).
\]
Applying boundary conditions:
\begin{itemize}
    \item At $x=0$: $\psi(0) = C_1 = 0$ $\implies \psi(x) = C_2 \sin(kx)$.
    \item At $x=L$: $\psi(L) = C_2 \sin(kL) = 0$. For $C_2 \neq 0$, $kL = n\pi$ ($n = 1, 2, 3, \dots$).
\end{itemize}
Thus, the allowed wavefunctions are:
\[
\psi_n(x) = \sqrt{\frac{2}{L}} \sin\left(\frac{n\pi x}{L}\right),
\]
where the prefactor ensures normalization.

\section*{Energy Eigenvalues}
The quantized energies are:
\[
E_n = \frac{n^2 \pi^2 \hbar^2}{2mL^2}, \quad n = 1, 2, 3, \dots.
\]
This quantization arises due to the boundary conditions and the discrete nature of $n$.

\section*{Expectation Values}
The expectation value of an operator $\hat{A}$ is:
\[
\langle A \rangle = \int_{-\infty}^\infty \psi^*(x) \hat{A} \psi(x) dx.
\]

\section*{Position Expectation Values}
For the ground state wavefunction $\psi_1(x) = \sqrt{\frac{2}{L}} \sin\left(\frac{\pi x}{L}\right)$:
\[
\langle x \rangle = \int_0^L x |\psi_1(x)|^2 dx.
\]
Substituting $|\psi_1(x)|^2 = \frac{2}{L} \sin^2\left(\frac{\pi x}{L}\right)$:
\[
\langle x \rangle = \frac{2}{L} \int_0^L x \sin^2\left(\frac{\pi x}{L}\right) dx.
\]
Using $\sin^2(\theta) = \frac{1}{2}(1 - \cos(2\theta))$:
\[
\langle x \rangle = \frac{1}{L} \int_0^L x \, dx - \frac{1}{L} \int_0^L x \cos\left(\frac{2\pi x}{L}\right) dx.
\]
The first term evaluates to:
\[
\int_0^L x \, dx = \frac{L^2}{2}.
\]
The second term vanishes due to symmetry. Thus:
\[
\langle x \rangle = \frac{L}{2}.
\]

\section*{Variance in Position}
The variance in $x$ is:
\[
(\Delta x)^2 = \langle x^2 \rangle - \langle x \rangle^2.
\]
Compute $\langle x^2 \rangle$:
\[
\langle x^2 \rangle = \frac{2}{L} \int_0^L x^2 \sin^2\left(\frac{\pi x}{L}\right) dx.
\]
Using integration by parts and trigonometric identities:
\[
\langle x^2 \rangle = \frac{L^2}{3} - \frac{L^2}{2\pi^2}.
\]
Thus:
\[
(\Delta x)^2 = \frac{L^2}{12} - \frac{L^2}{4}.
\]

\section*{Momentum Operator and Expectation Values}
The momentum operator is:
\[
\hat{p} = -i\hbar \frac{d}{dx}.
\]
For the ground state $\psi_1(x)$:
\[
\langle p \rangle = \int_0^L \psi_1^*(x) \hat{p} \psi_1(x) dx = 0,
\]
due to symmetry. For $\langle p^2 \rangle$:
\[
\langle p^2 \rangle = \int_0^L \psi_1^*(x) \hat{p}^2 \psi_1(x) dx = \frac{\pi^2 \hbar^2}{L^2}.
\]

\section*{Heisenberg Uncertainty Relation (HUR)}
The uncertainty in position and momentum are:
\[
\Delta x = \sqrt{\langle x^2 \rangle - \langle x \rangle^2}, \quad \Delta p = \sqrt{\langle p^2 \rangle - \langle p \rangle^2}.
\]
For the ground state:
\[
\Delta x \Delta p = \sqrt{\frac{L^2}{12}} \cdot \sqrt{\frac{\pi^2 \hbar^2}{L^2}} = \frac{\hbar}{2},
\]
satisfying the HUR:
\[
\Delta x \Delta p \geq \frac{\hbar}{2}.
\]

\section*{Conclusion}
The particle in a box demonstrates the quantization of energy and wavefunctions in quantum mechanics. The ground state satisfies the Heisenberg uncertainty relation, providing a clear example of the fundamental principles of quantum mechanics.

\section*{Euler's Formula and Trigonometric Identities}
Using Euler's formula:
\[
e^{i\theta} = \cos(\theta) + i\sin(\theta), \quad e^{-i\theta} = \cos(\theta) - i\sin(\theta),
\]
we derive the identities for $\cos(x)$ and $\sin(x)$:
\[
\cos(x) = \frac{e^{ix} + e^{-ix}}{2}, \quad \sin(x) = \frac{e^{ix} - e^{-ix}}{2i}.
\]
Steps:
\begin{itemize}
    \item Adding $e^{ix}$ and $e^{-ix}$:
    \[
    e^{ix} + e^{-ix} = 2\cos(x) \implies \cos(x) = \frac{e^{ix} + e^{-ix}}{2}.
    \]
    \item Subtracting $e^{-ix}$ from $e^{ix}$:
    \[
    e^{ix} - e^{-ix} = 2i\sin(x) \implies \sin(x) = \frac{e^{ix} - e^{-ix}}{2i}.
    \]
\end{itemize}

\section*{Solving the Boundary Value Problem}
Consider the differential equation:
\[
\psi''(x) + c\psi(x) = 0, \quad \psi(0) = \psi(L) = 0.
\]
\textbf{Case 1: $c > 0$}  
Let $c = k^2$ with $k > 0$. The equation becomes:
\[
\psi''(x) + k^2\psi(x) = 0.
\]
General solution:
\[
\psi(x) = A_1\sin(kx) + A_2\cos(kx).
\]
Boundary conditions:
\begin{itemize}
    \item At $x=0$: $\psi(0) = A_2 = 0 \implies \psi(x) = A_1\sin(kx)$.
    \item At $x=L$: $\psi(L) = A_1\sin(kL) = 0 \implies kL = n\pi \quad (n=1, 2, 3, \dots)$.
\end{itemize}
Thus:
\[
\psi_n(x) = B_n\sin\left(\frac{n\pi x}{L}\right).
\]

\textbf{Case 2: $c = 0$}  
The equation reduces to $\psi''(x) = 0$, giving:
\[
\psi(x) = Ax + B.
\]
Boundary conditions:
\[
\psi(0) = B = 0, \quad \psi(L) = A L = 0 \implies A = 0.
\]
Thus, $\psi(x) = 0$ is the only solution.

\textbf{Case 3: $c < 0$}  
Let $c = -k^2$ with $k > 0$. The equation becomes:
\[
\psi''(x) - k^2\psi(x) = 0.
\]
General solution:
\[
\psi(x) = A_1e^{kx} + A_2e^{-kx}.
\]
Boundary conditions:
\begin{itemize}
    \item At $x=0$: $\psi(0) = A_1 + A_2 = 0 \implies A_1 = -A_2$.
    \item At $x=L$: $\psi(L) = A_1e^{kL} + A_2e^{-kL} = 0$.
\end{itemize}
This condition implies $A_1 = A_2 = 0$, giving the trivial solution $\psi(x) = 0$.

\section*{Energy Levels of a Particle in a Box}
For a particle confined in a box of length $L$:
\[
E_n = \frac{n^2\pi^2\hbar^2}{2mL^2}, \quad n = 1, 2, 3, \dots.
\]
Difference between the first two energy levels:
\[
E_2 - E_1 = \frac{(2^2 - 1^2)\pi^2\hbar^2}{2mL^2} = \frac{3\pi^2\hbar^2}{2mL^2}.
\]

\section*{Heisenberg Uncertainty Relation (HUR)}
To verify $\Delta x \Delta p \geq \frac{\hbar}{2}$ for the ground state $\psi_1(x) = \sqrt{\frac{2}{L}} \sin\left(\frac{\pi x}{L}\right)$:
\[
\Delta A = \sqrt{\langle \hat{A}^2 \rangle - \langle \hat{A} \rangle^2}.
\]

\textbf{Position Uncertainty $\Delta x$:}
\[
\langle x \rangle = \int_0^L x |\psi_1(x)|^2 dx = \frac{L}{2}.
\]
\[
\langle x^2 \rangle = \int_0^L x^2 |\psi_1(x)|^2 dx = \frac{L^2}{3} - \frac{L^2}{2\pi^2}.
\]
\[
\Delta x = \sqrt{\langle x^2 \rangle - \langle x \rangle^2} = \sqrt{\frac{L^2}{12}}.
\]

\textbf{Momentum Uncertainty $\Delta p$:}
\[
\langle p \rangle = 0 \quad \text{(symmetry)}.
\]
\[
\langle p^2 \rangle = \frac{\pi^2\hbar^2}{L^2}.
\]
\[
\Delta p = \sqrt{\langle p^2 \rangle} = \frac{\pi\hbar}{L}.
\]

\textbf{HUR Verification:}
\[
\Delta x \Delta p = \sqrt{\frac{L^2}{12}} \cdot \frac{\pi\hbar}{L} = \frac{\pi\hbar}{\sqrt{12}} \geq \frac{\hbar}{2}.
\]

\end{multicols}
\end{document}
