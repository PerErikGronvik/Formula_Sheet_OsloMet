\documentclass[a4paper,12pt]{article}
\usepackage{multicol}
\usepackage{geometry}
\usepackage{amsmath}
\usepackage{amssymb}
\usepackage{titlesec} % To customize section title size and spacing
\geometry{a4paper, left=0.5in, right=0.5in, top=0.5in, bottom=0.5in}
\setlength{\parindent}{0pt} % Remove paragraph indentation
\setlength{\columnsep}{1cm} % Set spacing between columns

% Adjust section title font size and spacing
\titleformat{\section}
  {\normalfont\normalsize\bfseries} % Set section title font to small and bold
  {}{0pt}{}
\titlespacing*{\section}{0pt}{0.3cm}{0.2cm} % Adjust space before and after section

\begin{document}
\title{L11 - Finite Square Well Potential and Quantum Tunneling}
\author{}
\date{}
\maketitle

\begin{multicols}{2}

\section{Introduction to the Problem}
The finite square well potential is a classic problem in quantum mechanics. The potential \( U(x) \) is defined as:
\[
U(x) = 
\begin{cases} 
0, & |x| \leq \frac{L}{2}, \\ 
U_0, & |x| > \frac{L}{2}.
\end{cases}
\]
Here, \( U_0 > 0 \) is the height of the potential barrier, and \( L > 0 \) is the width of the well. The particle’s energy \( E \) can be less than \( U_0 \), making tunneling possible.

The Schrödinger equation (SE) for this system is:
\[
-\frac{d^2\psi}{dx^2} + U(x)\psi(x) = E\psi(x),
\]
where \( \psi(x) \) is the wavefunction.

\section{Energy Levels and Normalization}
The general solutions to the SE inside and outside the well are:
\[
\psi(x) = 
\begin{cases} 
B\cos(\beta x), & |x| \leq \frac{L}{2}, \\ 
Ae^{-\alpha|x|}, & |x| > \frac{L}{2}.
\end{cases}
\]
Here:
\[
\beta^2 = 2mE, \quad \alpha^2 = 2m(U_0 - E).
\]

The wavefunction must be continuous at \( x = \pm\frac{L}{2} \), which gives boundary conditions for matching the solutions.

\section{Continuity at \( x = \frac{L}{2} \)}
From the condition \( \psi(x) \) and \( \frac{d\psi}{dx} \) are continuous at \( x = \frac{L}{2} \):
\[
B\cos\left(\beta \frac{L}{2}\right) = A e^{-\alpha \frac{L}{2}},
\]
\[
-\beta B\sin\left(\beta \frac{L}{2}\right) = -\alpha A e^{-\alpha \frac{L}{2}}.
\]
Dividing these equations eliminates \( A \) and gives the transcendental equation for energy \( E \):
\[
\beta \tan\left(\beta \frac{L}{2}\right) = \alpha.
\]

\section{Energy Quantization}
The transcendental equation couples \( \beta \) and \( \alpha \), which depend on \( E \). Solving numerically gives the energy eigenvalues.

\subsection{Ground State Energy}
The lowest energy, or ground state energy, satisfies:
\[
E = \frac{\pi^2}{16}.
\]

\section{Wavefunction Normalization}
To normalize \( \psi(x) \), we require:
\[
\int_{-\infty}^\infty |\psi(x)|^2 dx = 1.
\]
Because \( \psi(x) \) is even, this simplifies to:
\[
2\int_0^\infty |\psi(x)|^2 dx = 1.
\]

For \( |x| \leq \frac{L}{2} \):
\[
\int_0^{L/2} B^2\cos^2(\beta x) dx = B^2 \frac{L}{2}.
\]

For \( |x| > \frac{L}{2} \):
\[
\int_{L/2}^\infty A^2 e^{-2\alpha x} dx = \frac{A^2}{2\alpha}.
\]

Equating these terms and solving for \( B \) gives:
\[
B = \sqrt{\frac{\pi}{\pi + 4}}.
\]

\section{Quantum Tunneling}
Quantum mechanics allows the particle to tunnel into the classically forbidden region \( |x| > \frac{L}{2} \). The probability of finding the particle in this region is:
\[
P(x \in [\frac{L}{2}, \infty]) = \int_{L/2}^\infty A^2 e^{-2\alpha x} dx > 0.
\]
This probability is non-zero, even though \( E < U_0 \), illustrating the tunneling effect.

\section{Expectation Value of Energy}
The expectation value of energy is calculated as:
\[
\langle E \rangle = \int_{-\infty}^\infty \psi^*(x) \left(-\frac{d^2}{dx^2} + U(x)\right) \psi(x) dx.
\]
Breaking this into kinetic and potential energy terms:
\[
\langle E \rangle = \langle T \rangle + \langle V \rangle,
\]
where:
\[
\langle T \rangle = \int_{-\infty}^\infty |\psi'(x)|^2 dx,
\]
\[
\quad
\langle V \rangle = \int_{-\infty}^\infty U(x)|\psi(x)|^2 dx.
\]

Detailed calculations confirm:
\[
\langle E \rangle = \frac{\pi^2}{16}.
\]

\section{Conclusion}
The finite square well demonstrates key quantum mechanics concepts:
\begin{itemize}
\item Energy quantization: Only specific energies are allowed.
\item Tunneling: Non-zero probability of being in the classically forbidden region.
\item Normalization ensures the physicality of the wavefunction.
\end{itemize}
These results align with fundamental quantum principles and provide insight into the behavior of particles in potential wells.


\section{Introduction}
The finite square well potential is a fundamental problem in quantum mechanics that illustrates concepts like energy quantization and quantum tunneling. The potential \( U(x) \) is defined as:
\[
U(x) = 
\begin{cases} 
0, & |x| \leq \frac{L}{2}, \\ 
U_0, & |x| > \frac{L}{2}.
\end{cases}
\]
Here:
\begin{itemize}
    \item \( U_0 > 0 \): Height of the potential barrier.
    \item \( L \): Width of the well.
\end{itemize}

We solve the time-independent Schrödinger equation (SE) for this potential:
\[
-\frac{d^2\psi}{dx^2} + U(x)\psi(x) = E\psi(x),
\]
where \( E \) is the energy of the particle. The particle is bound if \( E < U_0 \).

\section{Wavefunction and Boundary Conditions}
The general solution to the SE depends on the region:
\[
\psi(x) = 
\begin{cases} 
B\cos(\beta x), & |x| \leq \frac{L}{2}, \\ 
Ae^{-\alpha|x|}, & |x| > \frac{L}{2}.
\end{cases}
\]
The parameters \( \beta \) and \( \alpha \) are defined as:
\[
\beta^2 = 2mE, \quad \alpha^2 = 2m(U_0 - E).
\]
To find \( \psi(x) \), the wavefunction must satisfy:
\begin{itemize}
    \item Continuity of \( \psi(x) \) and its derivative \( \psi'(x) \) at \( x = \pm \frac{L}{2} \).
    \item Normalization: \( \int_{-\infty}^\infty |\psi(x)|^2 dx = 1 \).
\end{itemize}

\section{Matching Conditions at \( x = \frac{L}{2} \)}
At \( x = \frac{L}{2} \), the continuity conditions give:
\[
B\cos\left(\beta \frac{L}{2}\right) = A e^{-\alpha \frac{L}{2}},
\]
\[
-\beta B\sin\left(\beta \frac{L}{2}\right) = -\alpha A e^{-\alpha \frac{L}{2}}.
\]
Dividing these equations to eliminate \( A \) yields the transcendental equation:
\[
\beta \tan\left(\beta \frac{L}{2}\right) = \alpha.
\]
This equation relates \( E \) (via \( \beta \) and \( \alpha \)) to the potential parameters \( U_0 \) and \( L \).

\section{Energy Quantization}
The transcendental equation must be solved numerically to find the allowed energy levels \( E \). For the lowest energy state (ground state):
\[
E = \frac{\pi^2}{16}.
\]

\section{Normalization of the Wavefunction}
To normalize \( \psi(x) \), we enforce:
\[
\int_{-\infty}^\infty |\psi(x)|^2 dx = 1.
\]
Because \( \psi(x) \) is even, this simplifies to:
\[
2\int_0^\infty |\psi(x)|^2 dx = 1.
\]

\subsection{Inside the Well: \( |x| \leq \frac{L}{2} \)}
\[
\int_0^{L/2} B^2\cos^2(\beta x) dx = \frac{B^2}{2}\int_0^{L/2} (1 + \cos(2\beta x)) dx.
\]
Carrying out the integration:
\[
\int_0^{L/2} \cos(2\beta x) dx = \frac{\sin(\beta L)}{2\beta}.
\]

\subsection{Outside the Well: \( |x| > \frac{L}{2} \)}
\[
\int_{L/2}^\infty A^2 e^{-2\alpha x} dx = \frac{A^2}{2\alpha} e^{-2\alpha \frac{L}{2}}.
\]

Equating the contributions inside and outside gives the normalization constants:
\[
B = \sqrt{\frac{\pi}{\pi + 4}}, \quad A = B \frac{e^{\pi/4}}{\sqrt{2}}.
\]

\section{Tunneling Probability}
Quantum mechanics predicts that the particle can be found in the classically forbidden region \( |x| > \frac{L}{2} \). The probability is:
\[
P(x > \frac{L}{2}) = \int_{L/2}^\infty |\psi(x)|^2 dx.
\]
Substituting \( \psi(x) = A e^{-\alpha x} \):
\[
P(x > \frac{L}{2}) = \frac{A^2}{2\alpha} e^{-2\alpha \frac{L}{2}}.
\]
After simplification:
\[
P = \frac{1}{\pi + 4}.
\]

\section{Expectation Value of Energy}
The expectation value of energy is given by:
\[
\langle E \rangle = \int_{-\infty}^\infty \psi^*(x)\left(-\frac{d^2}{dx^2} + U(x)\right)\psi(x) dx.
\]
Breaking it into kinetic and potential energy:
\[
\langle E \rangle = \langle T \rangle + \langle V \rangle.
\]

\subsection{Kinetic Energy}
\[
\langle T \rangle = \int_{-\infty}^\infty |\psi'(x)|^2 dx.
\]
\subsection{Potential Energy}
\[
\langle V \rangle = \int_{-\infty}^\infty U(x)|\psi(x)|^2 dx.
\]
Detailed calculations confirm:
\[
\langle E \rangle = \frac{\pi^2}{16}.
\]

\section{Summary}
The finite square well potential demonstrates:
\begin{itemize}
    \item Energy quantization: Allowed energies are discrete.
    \item Tunneling: The particle has a non-zero probability of being in the forbidden region.
    \item Wavefunction normalization ensures \( \psi(x) \) is physical.
\end{itemize}

\end{multicols}
\end{document}
