\documentclass[a4paper,12pt]{article}
\usepackage{multicol}
\usepackage{geometry}
\usepackage{amsmath}
\usepackage{amssymb}
\usepackage{titlesec} % To customize section title size and spacing
\geometry{a4paper, left=0.5in, right=0.5in, top=0.5in, bottom=0.5in}
\setlength{\parindent}{0pt} % Remove paragraph indentation
\setlength{\columnsep}{1cm} % Set spacing between columns

% Adjust section title font size and spacing
\titleformat{\section}
  {\normalfont\normalsize\bfseries} % Set section title font to small and bold
  {}{0pt}{}
\titlespacing*{\section}{0pt}{0.3cm}{0.2cm} % Adjust space before and after section

\begin{document}
\title{L10 - Finite Potential Well and Bound States}
\author{}
\date{}
\maketitle
\begin{multicols}{2}

\section{Introduction}
We analyze the finite potential square well to determine bound state solutions. The potential is given as:
\[
U(x) = 
\begin{cases} 
U_0 & \text{if } x < -L/2 \text{ or } x > L/2, \\
0 & \text{if } -L/2 \leq x \leq L/2,
\end{cases}
\]
where \( U_0 > 0 \) is the potential barrier, and \( L \) is the width of the well. We solve the Schrödinger equation:
\[
-\frac{\hbar^2}{2m} \frac{d^2\psi(x)}{dx^2} + U(x)\psi(x) = E\psi(x),
\]
for bound states (\( E < U_0 \)).

\section{Regions of the Potential Well}
The potential well is divided into three regions:

\begin{itemize}
    \item \textbf{Region I:} \( x < -L/2 \), where \( U(x) = U_0 \).
    \item \textbf{Region II:} \( -L/2 \leq x \leq L/2 \), where \( U(x) = 0 \).
    \item \textbf{Region III:} \( x > L/2 \), where \( U(x) = U_0 \).
\end{itemize}

\subsection{Wavefunctions in Each Region}
The wavefunctions in each region are solved as follows:

\textbf{Region I and III:} For \( x < -L/2 \) and \( x > L/2 \), the Schrödinger equation becomes:
\[
-\frac{\hbar^2}{2m} \frac{d^2\psi}{dx^2} + U_0\psi = E\psi.
\]
Rewriting, 
\[
\frac{d^2\psi}{dx^2} = \frac{2m}{\hbar^2}(U_0 - E)\psi = \alpha^2 \psi,
\]
where \( \alpha^2 = \frac{2m(U_0 - E)}{\hbar^2} \). The general solution is:
\[
\psi(x) = A e^{\alpha x} + B e^{-\alpha x},
\]
but we discard the exponentially growing terms at infinity, leaving:
\[
\psi(x) = B e^{\alpha x} \text{ for } x < -L/2, \quad \psi(x) = A e^{-\alpha x} \text{ for } x > L/2.
\]

\textbf{Region II:} For \( -L/2 \leq x \leq L/2 \), the Schrödinger equation becomes:
\[
-\frac{\hbar^2}{2m} \frac{d^2\psi}{dx^2} = E\psi.
\]
Rewriting,
\[
\frac{d^2\psi}{dx^2} = -\beta^2 \psi, \quad \beta^2 = \frac{2mE}{\hbar^2}.
\]
The general solution is:
\[
\psi(x) = C \cos(\beta x) + D \sin(\beta x).
\]

\section{Boundary Conditions}
The wavefunction \( \psi(x) \) and its derivative \( \psi'(x) \) must be continuous at \( x = \pm L/2 \). This gives four boundary conditions:
\[
\psi_I(-L/2) = \psi_{II}(-L/2), \quad \psi_{II}(L/2) = \psi_{III}(L/2),
\]
\[
\psi'_I(-L/2) = \psi'_{II}(-L/2), \quad \psi'_{II}(L/2) = \psi'_{III}(L/2).
\]

\section{Even and Odd Solutions}
Symmetry considerations allow us to classify the solutions as even or odd:

\textbf{Even Solutions:} \( \psi(-x) = \psi(x) \).
\[
\psi(x) = 
\begin{cases} 
B e^{\alpha x}, & x < -L/2, \\
C \cos(\beta x), & -L/2 \leq x \leq L/2, \\
A e^{-\alpha x}, & x > L/2.
\end{cases}
\]

\textbf{Odd Solutions:} \( \psi(-x) = -\psi(x) \).
\[
\psi(x) = 
\begin{cases} 
B e^{\alpha x}, & x < -L/2, \\
D \sin(\beta x), & -L/2 \leq x \leq L/2, \\
-A e^{-\alpha x}, & x > L/2.
\end{cases}
\]

\section{Matching Conditions}
Using the boundary conditions, we find the transcendental equations for \( \beta \) and \( \alpha \):

\textbf{Even States:}
\[
\beta \tan\left(\frac{\beta L}{2}\right) = \alpha,
\]
\textbf{Odd States:}
\[
\beta \cot\left(\frac{\beta L}{2}\right) = \alpha,
\]
with the constraint:
\[
\alpha^2 + \beta^2 = \frac{2mU_0}{\hbar^2}.
\]

\section{Energy Quantization}
The transcendental equations for \( \beta \) and \( \alpha \) determine the allowed energy levels \( E \) for each state. Energy is related to \( \beta \) by:
\[
E = \frac{\hbar^2 \beta^2}{2m}.
\]

\section{Limiting Case: Infinite Square Well}
For \( U_0 \to \infty \), \( \alpha \to \infty \), and the transcendental equations reduce to the boundary conditions of the infinite square well. The energy levels are:
\[
E_n = \frac{n^2 \pi^2 \hbar^2}{2mL^2}, \quad n = 1, 2, 3, \dots
\]

\section{Graphical Interpretation}
The solutions can be visualized by plotting \( \beta \tan(\beta L/2) \) (even states) and \( \beta \cot(\beta L/2) \) (odd states) against \( \alpha^2 + \beta^2 = \rho^2 \), where \( \rho^2 = \frac{2mU_0}{\hbar^2} \). The intersections represent the allowed energy levels.

\section{Overview of the Finite Square Well}
The finite square potential well is a system where a particle is confined by a potential \( U(x) \) defined as:
\[
U(x) = 
\begin{cases} 
U_0, & |x| > L/2, \\
0, & |x| \leq L/2,
\end{cases}
\]
where \( U_0 > 0 \) is the depth of the well, and \( L \) is the width of the well. The Schrödinger equation for this system is:
\[
-\frac{\hbar^2}{2m} \frac{d^2\psi(x)}{dx^2} + U(x)\psi(x) = E\psi(x),
\]
where \( \psi(x) \) is the wavefunction, \( E \) is the energy, and \( m \) is the mass of the particle.

For bound states (\( E < U_0 \)), the wavefunction is confined within the well and decays exponentially outside.

\section{Regions of the Potential Well}
We divide the system into three regions:

\textbf{Region I:} \( x < -L/2 \), where \( U(x) = U_0 \).

\textbf{Region II:} \( -L/2 \leq x \leq L/2 \), where \( U(x) = 0 \).

\textbf{Region III:} \( x > L/2 \), where \( U(x) = U_0 \).

\subsection{Wavefunctions in Each Region}
\textbf{Region I and III:} For \( x < -L/2 \) and \( x > L/2 \), the Schrödinger equation becomes:
\[
-\frac{\hbar^2}{2m} \frac{d^2\psi}{dx^2} + U_0\psi = E\psi.
\]
Simplifying:
\[
\frac{d^2\psi}{dx^2} = \frac{2m}{\hbar^2}(U_0 - E)\psi = \alpha^2 \psi,
\]
where \( \alpha^2 = \frac{2m(U_0 - E)}{\hbar^2} \). The general solution is:
\[
\psi(x) = A e^{\alpha x} + B e^{-\alpha x}.
\]
Since \( \psi(x) \) must decay as \( x \to \pm\infty \), we eliminate the growing exponential terms, leaving:
\[
\psi_I(x) = B e^{\alpha x}, \quad \psi_{III}(x) = A e^{-\alpha x}.
\]

\textbf{Region II:} For \( -L/2 \leq x \leq L/2 \), the Schrödinger equation becomes:
\[
-\frac{\hbar^2}{2m} \frac{d^2\psi}{dx^2} = E\psi.
\]
Simplifying:
\[
\frac{d^2\psi}{dx^2} = -\beta^2 \psi, \quad \beta^2 = \frac{2mE}{\hbar^2}.
\]
The general solution is:
\[
\psi_{II}(x) = C \cos(\beta x) + D \sin(\beta x).
\]

\section{Symmetry of the Solutions}
Due to the symmetry of \( U(x) \), the solutions can be classified as:

\textbf{Even States:} \( \psi(-x) = \psi(x) \).  
\textbf{Odd States:} \( \psi(-x) = -\psi(x) \).

\subsection{Wavefunctions for Even States}
For even states, the wavefunction has the form:
\[
\psi(x) = 
\begin{cases} 
B e^{\alpha x}, & x < -L/2, \\
C \cos(\beta x), & -L/2 \leq x \leq L/2, \\
A e^{-\alpha x}, & x > L/2.
\end{cases}
\]

\subsection{Wavefunctions for Odd States}
For odd states, the wavefunction has the form:
\[
\psi(x) = 
\begin{cases} 
B e^{\alpha x}, & x < -L/2, \\
D \sin(\beta x), & -L/2 \leq x \leq L/2, \\
-A e^{-\alpha x}, & x > L/2.
\end{cases}
\]

\section{Boundary Conditions}
The wavefunction \( \psi(x) \) and its derivative \( \psi'(x) \) must be continuous at \( x = \pm L/2 \). This gives four conditions:

1. \( \psi_I(-L/2) = \psi_{II}(-L/2) \),
2. \( \psi'_I(-L/2) = \psi'_{II}(-L/2) \),
3. \( \psi_{II}(L/2) = \psi_{III}(L/2) \),
4. \( \psi'_{II}(L/2) = \psi'_{III}(L/2) \).

\subsection{Transcendental Equations}
From the boundary conditions, we derive the following equations:

\textbf{Even States:}
\[
\beta \tan\left(\frac{\beta L}{2}\right) = \alpha.
\]

\textbf{Odd States:}
\[
\beta \cot\left(\frac{\beta L}{2}\right) = \alpha.
\]

These transcendental equations must be solved numerically or graphically to find the allowed \( \beta \) values.

\section{Normalization Condition}
The total probability must equal 1:
\[
\int_{-\infty}^\infty |\psi(x)|^2 dx = 1.
\]
For even states:
\[
1 = 2\left( \frac{e^{-\alpha L}}{\alpha}|A|^2 + \frac{L}{2}|C|^2 + \frac{\sin(\beta L)}{2\beta}|C|^2 \right).
\]

For odd states, normalization follows a similar approach.

\section{Number of Bound States}
The number of bound states is determined graphically by solving:
\[
x_1^2 + x_2^2 = R^2,
\]
where \( x_1 = \beta L/2 \), \( x_2 = \alpha L/2 \), and \( R^2 = \frac{2mU_0 L^2}{\hbar^2 4} \). The intersections with \( x_2 = x_1 \tan(x_1) \) (even) and \( x_2 = -x_1 \cot(x_1) \) (odd) give the bound states.

\section{Special Case: Infinite Square Well}
When \( U_0 \to \infty \), \( \alpha \to \infty \), and only solutions for \( \beta \) remain. The energy levels reduce to:
\[
E_n = \frac{n^2 \pi^2 \hbar^2}{2mL^2}, \quad n = 1, 2, 3, \dots
\]
\end{multicols}
\end{document}
