\documentclass[a4paper,12pt]{article}
\usepackage{multicol}
\usepackage{geometry}
\usepackage{amsmath}
\usepackage{amssymb}
\usepackage{titlesec} % To customize section title size and spacing
\geometry{a4paper, left=0.5in, right=0.5in, top=0.5in, bottom=0.5in}
\setlength{\parindent}{0pt} % Remove paragraph indentation
\setlength{\columnsep}{1cm} % Set spacing between columns

% Adjust section title font size and spacing
\titleformat{\section}
  {\normalfont\normalsize\bfseries} % Set section title font to small and bold
  {}{0pt}{}
\titlespacing*{\section}{0pt}{0.3cm}{0.2cm} % Adjust space before and after section

\begin{document}
\title{L6 - Introduction to Quantum Mechanics}
\author{}
\date{}
\maketitle
\begin{multicols}{2}

\section*{Wave-Particle Duality}
Quantum mechanics introduces the concept of wave-particle duality, where particles like electrons and photons exhibit both wave-like and particle-like properties:
\begin{itemize}
    \item \textbf{Wave-like:} Interference, diffraction.
    \item \textbf{Particle-like:} Momentum, collisions.
\end{itemize}

For example:
\begin{itemize}
    \item Light behaves as a wave (diffraction) and as particles (photoelectric effect).
    \item Electrons exhibit wave-like behavior (interference patterns).
\end{itemize}

\section*{The Failure of Classical Mechanics}
Classical mechanics cannot explain:
\begin{itemize}
    \item Atomic stability: Atoms in classical mechanics would collapse as electrons radiate energy continuously.
    \item Discrete energy levels: Classical theory predicts a continuous spectrum of energies.
\end{itemize}
Quantum mechanics resolves this with quantized energy states, ensuring stability.

\section*{Planck’s Constant and Photon Energy}
The energy of a photon is given by:
\[
E = hf = \frac{hc}{\lambda},
\]
where:
\begin{itemize}
    \item $h = 6.626 \times 10^{-34} \, \mathrm{Js}$ (Planck’s constant),
    \item $f$ = frequency,
    \item $\lambda$ = wavelength,
    \item $c = 3 \times 10^8 \, \mathrm{m/s}$ (speed of light).
\end{itemize}

The momentum of a photon is:
\[
p = \frac{h}{\lambda}.
\]

\section*{The Photoelectric Effect}
The photoelectric effect demonstrates the particle nature of light:
\begin{align*}
K_\text{max} &= \frac{1}{2} mv_\text{max}^2 = eU_0, \\
hf &= \phi + K_\text{max},
\end{align*}
where:
\begin{itemize}
    \item $K_\text{max}$ = maximum kinetic energy of ejected electrons,
    \item $\phi$ = work function (energy needed to eject an electron),
    \item $U_0$ = stopping potential.
\end{itemize}
Rewriting, we find:
\[
eU_0 = hf - \phi.
\]

The slope of the graph of $U_0$ vs. $f$ is $\frac{h}{e}$.

\section*{The Schrödinger Equation}
Quantum systems are described by a wavefunction $\psi(x,t)$, which satisfies the Schrödinger equation:
\[
i\hbar \frac{\partial \psi(x,t)}{\partial t} = \hat{H} \psi(x,t),
\]
where:
\begin{itemize}
    \item $\hbar = \frac{h}{2\pi}$ is the reduced Planck constant,
    \item $\hat{H}$ is the Hamiltonian operator, representing the total energy:
    \[
    \hat{H} = \hat{T} + \hat{U},
    \]
    with $\hat{T}$ the kinetic energy operator and $\hat{U}$ the potential energy operator.
\end{itemize}

\section*{Kinetic and Potential Energy Operators}
\begin{itemize}
    \item Kinetic energy operator:
    \[
    \hat{T} = -\frac{\hbar^2}{2m} \frac{\partial^2}{\partial x^2}.
    \]
    \item Potential energy operator:
    \[
    \hat{U} = U(x),
    \]
    which acts as a multiplication operator.
\end{itemize}

Thus, the Hamiltonian becomes:
\[
\hat{H} = -\frac{\hbar^2}{2m} \frac{\partial^2}{\partial x^2} + U(x).
\]

\section*{Time-Independent Schrödinger Equation}
For stationary states, $\psi(x,t)$ can be separated:
\[
\psi(x,t) = \psi(x)\phi(t).
\]
Substituting into the time-dependent equation, we find:
\[
i\hbar \frac{\phi'(t)}{\phi(t)} = -\frac{\hbar^2}{2m} \frac{\psi''(x)}{\psi(x)} + U(x).
\]

As the left-hand side depends only on $t$ and the right-hand side only on $x$, both must equal a constant $E$. This gives:
\begin{align*}
\text{(A)} & \quad i\hbar \frac{\phi'(t)}{\phi(t)} = E, \quad \phi(t) = Ce^{-\frac{iEt}{\hbar}}, \\
\text{(B)} & \quad -\frac{\hbar^2}{2m} \psi''(x) + U(x)\psi(x) = E\psi(x).
\end{align*}

Equation (B) is the time-independent Schrödinger equation.

\section*{Expectation Values}
In quantum mechanics, physical quantities correspond to operators. The expectation value of an operator $\hat{A}$ is:
\[
\langle A \rangle = \int_{-\infty}^\infty \psi^*(x) \hat{A} \psi(x) \, dx,
\]
where $\psi^*(x)$ is the complex conjugate of $\psi(x)$.

\section*{Examples of Expectation Values}
\begin{itemize}
    \item Position:
    \[
    \langle x \rangle = \int_{-\infty}^\infty x |\psi(x)|^2 \, dx.
    \]
    \item Momentum:
    \[
    \langle p \rangle = \int_{-\infty}^\infty \psi^*(x) \left(-i\hbar \frac{\partial}{\partial x}\right) \psi(x) \, dx.
    \]
    \item Energy:
    \[
    \langle E \rangle = \int_{-\infty}^\infty \psi^*(x) \hat{H} \psi(x) \, dx.
    \]
\end{itemize}

\section*{Normalization Condition}
The wavefunction $\psi(x)$ must satisfy:
\[
\int_{-\infty}^\infty |\psi(x)|^2 \, dx = 1,
\]
ensuring the total probability is 1.

\section*{Divergence of the Integral for a Monotonic Wavefunction}
Consider a wavefunction $\psi(x)$ defined on the real line, where $\psi(x)$ is:
\begin{itemize}
    \item Monotonic (not oscillating) and continuous for $x > x_0 > 0$, and
    \item Does not approach zero as $x \to +\infty$.
\end{itemize}

We aim to show:
\[
\int_{-\infty}^\infty |\psi(x)|^2 dx = \infty.
\]

\subsection*{Step-by-step Solution}
Since $\psi(x)$ is monotonic and does not approach zero as $x \to +\infty$, there exists a constant $\epsilon > 0$ such that $|\psi(x)|^2 > \epsilon$ for all $x > x_0$. This gives:
\[
\int_{x_0}^\infty |\psi(x)|^2 dx \geq \int_{x_0}^\infty \epsilon \, dx.
\]

Calculating the right-hand integral:
\[
\int_{x_0}^\infty \epsilon \, dx = \epsilon \int_{x_0}^\infty 1 \, dx = \epsilon \left[ x \right]_{x_0}^\infty = \epsilon (\infty - x_0) = \infty.
\]

Thus:
\[
\int_{-\infty}^\infty |\psi(x)|^2 dx = \infty,
\]
proving the result. This condition implies $\psi(x)$ cannot represent a physically normalized wavefunction.

\section*{Photoelectric Effect on Silver Surface}
In a photoelectric effect experiment, ultraviolet light with frequency $f = 1.1 \times 10^{15} \, \mathrm{Hz}$ is shone on a silver surface. The stopping potential is $U_0 = 1.5 \, \mathrm{V}$.

\subsection*{Objective}
\begin{enumerate}
    \item Calculate the energy of the incident photons in electron volts (eV).
    \item Determine the work function of silver.
    \item Verify that the work function corresponds to known values for silver.
\end{enumerate}

\subsection*{Constants}
\begin{itemize}
    \item Planck's constant: $h = 6.63 \times 10^{-34} \, \mathrm{J \cdot s}$,
    \item Elementary charge: $e = 1.6 \times 10^{-19} \, \mathrm{C}$,
    \item Speed of light: $c = 3.0 \times 10^8 \, \mathrm{m/s}$.
\end{itemize}

\subsection*{Solution}
\subsubsection*{(a) Energy of Incident Photons}
The energy of a photon is given by:
\[
E = hf.
\]

Substituting the known values:
\[
E = (6.63 \times 10^{-34} \, \mathrm{J \cdot s})(1.1 \times 10^{15} \, \mathrm{Hz}) = 7.293 \times 10^{-19} \, \mathrm{J}.
\]

Convert to eV:
\[
E = \frac{7.293 \times 10^{-19} \, \mathrm{J}}{1.6 \times 10^{-19} \, \mathrm{J/eV}} = 4.56 \, \mathrm{eV}.
\]

\subsubsection*{(b) Determining the Work Function}
The kinetic energy of the ejected electrons is related to the stopping potential $U_0$ by:
\[
K = eU_0.
\]

Substituting values:
\[
K = (1.6 \times 10^{-19} \, \mathrm{C})(1.5 \, \mathrm{V}) = 2.4 \times 10^{-19} \, \mathrm{J}.
\]

Convert to eV:
\[
K = \frac{2.4 \times 10^{-19} \, \mathrm{J}}{1.6 \times 10^{-19} \, \mathrm{J/eV}} = 1.5 \, \mathrm{eV}.
\]

The work function $\phi$ is then given by:
\[
\phi = E - K.
\]

Substituting values:
\[
\phi = 4.56 \, \mathrm{eV} - 1.5 \, \mathrm{eV} = 4.2 \, \mathrm{eV}.
\]

\subsubsection*{(c) Identifying the Material}
The calculated work function $\phi = 4.2 \, \mathrm{eV}$ matches the known work function of silver. Therefore, the material is confirmed to be silver.

\section*{Key Equations Recap}
\begin{itemize}
    \item Photon energy: 
    \[
    E = hf = \frac{hc}{\lambda}.
    \]
    \item Kinetic energy of ejected electrons:
    \[
    K = eU_0.
    \]
    \item Work function:
    \[
    \phi = E - K.
    \]
\end{itemize}

\end{multicols}
\end{document}
