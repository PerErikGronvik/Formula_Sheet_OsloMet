\documentclass[a4paper,12pt]{article}
\usepackage{multicol}
\usepackage{geometry}
\usepackage{amsmath}
\usepackage{amssymb}
\usepackage{titlesec}
\geometry{a4paper, left=0.5in, right=0.5in, top=0.5in, bottom=0.5in}
\setlength{\parindent}{0pt}
\setlength{\columnsep}{1cm}

\titleformat{\section}
  {\normalfont\normalsize\bfseries}
  {}{0pt}{}
\titlespacing*{\section}{0pt}{0.3cm}{0.2cm}

\begin{document}
\title{\textbf{L9 - QM - Operators and Eigenfunctions - Fourier Expansion }}
\author{}
\date{}
\maketitle

\begin{multicols}{2}

\section*{1. Hermitian Operators and Eigenfunctions}
Let $\hat{H}$ be a Hermitian Hamiltonian operator. For any wavefunctions $\Phi$ and $\Psi$, the Hermitian property ensures:
\[
\langle \Phi, \hat{H} \Psi \rangle = \langle \hat{H} \Phi, \Psi \rangle.
\]

If $\{\psi_n\}_{n=1}^\infty$ is an orthonormal set of eigenfunctions of $\hat{H}$, then:
\[
\langle \psi_n, \psi_m \rangle = \delta_{nm},
\]
where $\delta_{nm}$ is the Kronecker delta. Each eigenfunction $\psi_n$ corresponds to an eigenvalue $E_n$ such that:
\[
\hat{H} \psi_n = E_n \psi_n.
\]

\section*{2. Expanding Admissible Wavefunctions}
Any admissible wavefunction $\Phi(x)$ can be expressed as a linear combination of eigenfunctions:
\[
\Phi(x) = \sum_{n=1}^\infty c_n \psi_n(x), \quad \text{where } c_n = \langle \psi_n, \Phi \rangle.
\]

\subsection*{Probability Interpretation}
The coefficient $|c_n|^2$ represents the probability of measuring the energy $E_n$.

\section*{3. Expectation Value of Energy}
The expectation value of energy in the state $\Phi$ is given by:
\[
\langle E \rangle = \langle \Phi, \hat{H} \Phi \rangle.
\]
Expanding $\Phi(x)$ in terms of $\{\psi_n(x)\}$, we have:
\[
\Phi(x) = \sum_{n=1}^\infty c_n \psi_n(x), \quad \Phi^*(x) = \sum_{n=1}^\infty c_n^* \psi_n(x).
\]
Substituting:
\[
\langle E \rangle = \int \Phi^*(x) \hat{H} \Phi(x) \, dx.
\]
Using the orthonormality and eigenvalue relations:
\[
\langle E \rangle = \sum_{n=1}^\infty |c_n|^2 E_n.
\]

\section*{4. Fourier Expansion in a Box}
Consider a particle in a 1D box of length $L$ with wavefunctions:
\[
\psi_n(x) = \sqrt{\frac{2}{L}} \sin\left(\frac{n\pi x}{L}\right), \quad n = 1, 2, 3, \dots
\]
and eigenvalues:
\[
E_n = \frac{n^2 \hbar^2 \pi^2}{2mL^2}.
\]

\subsection*{Expanding $\Phi(x)$ in $\{\psi_n(x)\}$}
Given $\Phi(x)$ on $[0, L]$, write:
\[
\Phi(x) = \sum_{n=1}^\infty c_n \psi_n(x), \quad c_n = \langle \psi_n, \Phi \rangle.
\]
Substitute:
\[
c_n = \sqrt{\frac{2}{L}} \int_0^L \Phi(x) \sin\left(\frac{n\pi x}{L}\right) dx.
\]

\section*{5. Example: $\Phi(x) = \sqrt{30} x(1-x)$ on $[0,1]$}
Normalize $\Phi(x)$:
\[
\int_0^1 |\Phi(x)|^2 dx = 1 \implies C = \sqrt{30}.
\]
Find $c_n$:
\[
c_n = \sqrt{2} \int_0^1 \sqrt{30} x(1-x) \sin(n\pi x) dx.
\]
Split the integral:
\[
c_n = \sqrt{60} \left[ \int_0^1 x \sin(n\pi x) dx - \int_0^1 x^2 \sin(n\pi x) dx \right].
\]

\subsection*{Computing the Integrals}
For the first integral:
\[
\int_0^1 x \sin(n\pi x) dx =
\]
\[
\left[-\frac{x \cos(n\pi x)}{n\pi} + \frac{\sin(n\pi x)}{(n\pi)^2}\right]_0^1.
\]
At $x=1$, $\sin(n\pi) = 0$, $\cos(n\pi) = (-1)^n$. Substituting:
\[
\int_0^1 x \sin(n\pi x) dx = \frac{(-1)^n}{n\pi}.
\]

For the second integral:
\[
\int_0^1 x^2 \sin(n\pi x) \, dx =
\]
\[
\left[
    -\frac{x^2 \cos(n\pi x)}{n\pi} 
    + \frac{2x \sin(n\pi x)}{(n\pi)^2}
    - \frac{2\cos(n\pi x)}{(n\pi)^3}
\right]_0^1.
\]



At $x=1$, simplify using $\sin(n\pi)=0$. Substituting:
\[
\int_0^1 x^2 \sin(n\pi x) dx = -\frac{(-1)^n}{n\pi} + \frac{2(-1)^n}{(n\pi)^3}.
\]

Combine results:
\[
c_n = \sqrt{60} \left[\frac{(-1)^n}{n\pi} - \left(-\frac{(-1)^n}{n\pi} + \frac{2(-1)^n}{(n\pi)^3}\right)\right].
\]

\subsection*{Final Coefficients}
Simplify $c_n$:
\[
c_n = \begin{cases}
0 & \text{if } n \text{ is even}, \\
\frac{4\sqrt{60}}{n^3\pi^3} & \text{if } n \text{ is odd}.
\end{cases}
\]

\section*{6. Summary of Results}
The wavefunction $\Phi(x)$ is expanded as:
\[
\Phi(x) = \sum_{n=1}^\infty c_n \psi_n(x), \quad c_n = \frac{4\sqrt{60}}{n^3\pi^3} \text{ (for odd } n).
\]

\section*{7. Inner Products in Quantum Mechanics}
The inner product of two wavefunctions is defined as:
\[
\langle \psi, \phi \rangle = \int_{-\infty}^\infty \psi^*(x) \phi(x) dx.
\]
For the particle in a box:
\[
\langle \psi_n, \psi_m \rangle = \delta_{nm}.
\]
\section*{1. Minimum Energy of a Quantum System}
The total energy $E$ of a quantum system satisfies:
\[
E > \min_{x \in \mathbb{R}} U(x) = U_\text{min}.
\]
This is derived from the Schrödinger equation, which relates kinetic and potential energy.

\subsection*{1.1 Rewriting the Schrödinger Equation}
The time-independent Schrödinger equation for a particle of mass $m$ is:
\[
-\frac{\hbar^2}{2m} \psi''(x) + U(x)\psi(x) = E\psi(x).
\]
Rearrange terms:
\[
-\frac{\hbar^2}{2m} \psi''(x) = (E - U(x))\psi(x).
\]
Multiply by $\psi(x)$:
\[
\psi(x)\psi''(x) = \frac{2m}{\hbar^2} (U(x) - E) \psi^2(x).
\]

\subsection*{1.2 Integration Over All Space}
Integrate over all $x$:
\[
\int_{-\infty}^\infty \psi(x)\psi''(x) \, dx =
\]
\[
\frac{2m}{\hbar^2} \int_{-\infty}^\infty (U(x) - E)\psi^2(x) \, dx.
\]
The left-hand side can be simplified using integration by parts.

\subsection*{1.3 Left-Hand Side (Integration by Parts)}
Let $u = \psi(x)$ and $dv = \psi''(x) dx$. Then:
\[
\int \psi(x)\psi''(x) \, dx =
\]
\[
\left[\psi(x)\psi'(x)\right]_{-\infty}^\infty - \int \left(\psi'(x)\right)^2 dx.
\]
Assuming $\psi(x)$ and $\psi'(x)$ vanish at infinity (normalizable wavefunctions):
\[
\int \psi(x)\psi''(x) \, dx = - \int_{-\infty}^\infty \left(\psi'(x)\right)^2 dx.
\]

\subsection*{1.4 Relating to Kinetic Energy}
The expectation value of the kinetic energy operator $\hat{T}$ is:
\[
\langle \psi, \hat{T} \psi \rangle = \int_{-\infty}^\infty \psi^*(x) \left(-\frac{\hbar^2}{2m} \frac{d^2}{dx^2}\right) \psi(x) \, dx.
\]
Substituting $\hat{T}$:
\[
\langle \psi, \hat{T} \psi \rangle = \frac{\hbar^2}{2m} \int_{-\infty}^\infty \left(\psi'(x)\right)^2 dx.
\]
Thus:
\[
\int_{-\infty}^\infty \psi(x)\psi''(x) dx = -\frac{2m}{\hbar^2} \langle \psi, \hat{T} \psi \rangle.
\]

\subsection*{1.5 Substituting Back}
Substituting this into the integrated equation:
\[
-\frac{2m}{\hbar^2} \langle \psi, \hat{T} \psi \rangle = \frac{2m}{\hbar^2} \int_{-\infty}^\infty (U(x) - E)\psi^2(x) \, dx.
\]
Divide by $\frac{2m}{\hbar^2}$:
\[
-\langle \psi, \hat{T} \psi \rangle = \int_{-\infty}^\infty (U(x) - E)\psi^2(x) \, dx.
\]

\subsection*{1.6 Applying the Minimum of $U(x)$}
Since $U(x) \geq U_\text{min}$ for all $x$ and $\psi^2(x) \geq 0$, we have:
\[
(U(x) - E) \geq (U_\text{min} - E).
\]
Multiply by $\psi^2(x)$ and integrate:
\[
\int_{-\infty}^\infty (U(x) - E)\psi^2(x) \, dx \geq (U_\text{min} - 
\]
\[
E)\int_{-\infty}^\infty \psi^2(x) dx.
\]
Using normalization $\int_{-\infty}^\infty \psi^2(x) dx = 1$:
\[
-\langle \psi, \hat{T} \psi \rangle \geq U_\text{min} - E.
\]
Rearranging:
\[
E \geq U_\text{min} - \langle \psi, \hat{T} \psi \rangle.
\]

\subsection*{1.7 Conclusion}
The kinetic energy $\langle \psi, \hat{T} \psi \rangle$ is strictly positive, so:
\[
U_\text{min} - \langle \psi, \hat{T} \psi \rangle < U_\text{min}.
\]
Thus:
\[
E > U_\text{min}.
\]

\section*{2. Calculating Probability for a Specific Energy Level}
For a particle in a box, the energy eigenstates are given by:
\[
\psi_n(x) = \sqrt{\frac{2}{L}} \sin\left(\frac{n\pi x}{L}\right), \quad n = 1, 2, 3, \dots
\]
with energy levels:
\[
E_n = \frac{n^2 \hbar^2 \pi^2}{2mL^2}.
\]

\subsection*{2.1 Expansion Coefficient}
The wavefunction $\Phi(x)$ is expanded as:
\[
\Phi(x) = \sum_{n=1}^\infty c_n \psi_n(x), \quad c_n = \langle \psi_n, \Phi \rangle.
\]
For $n=1$:
\[
c_1 = \int_0^L \Phi(x) \sqrt{\frac{2}{L}} \sin\left(\frac{\pi x}{L}\right) dx.
\]

\subsection*{2.2 Probability for $E_1$}
The probability of measuring $E_1$ is:
\[
P(E = E_1) = |c_1|^2.
\]
Given $c_1 = -\frac{8\sqrt{30}}{\pi^{7/2}}$, compute:
\[
|c_1|^2 = \left(-\frac{8\sqrt{30}}{\pi^{7/2}}\right)^2 = \frac{64 \cdot 30}{\pi^7} = \frac{1920}{\pi^7}.
\]
Thus:
\[
P(E = E_1) = \frac{1920}{\pi^7}.
\]
\end{multicols}
\end{document}
