\documentclass[a4paper,12pt]{article}
\usepackage{multicol}
\usepackage{geometry}
\usepackage{amsmath}
\usepackage{amssymb}
\usepackage{titlesec} % To customize section title size and spacing
\geometry{a4paper, left=0.5in, right=0.5in, top=0.5in, bottom=0.5in}
\setlength{\parindent}{0pt} % Remove paragraph indentation
\setlength{\columnsep}{1cm} % Set spacing between columns

% Adjust section title font size and spacing
\titleformat{\section}
  {\normalfont\normalsize\bfseries} % Set section title font to small and bold
  {}{0pt}{}
\titlespacing*{\section}{0pt}{0.3cm}{0.2cm} % Adjust space before and after section

\begin{document}
\title{L13 - Time-Dependent Schrodinger Equation TDSE and Numerical Solutions}
\author{}
\date{}
\maketitle
\begin{multicols}{2}

\section{Introduction to the TDSE}
The time-dependent Schrödinger equation (TDSE) for a particle in one dimension is:
\[
i \hbar \frac{\partial \Psi(x,t)}{\partial t} = \left(-\frac{\hbar^2}{2m} \frac{\partial^2}{\partial x^2} + U(x)\right)\Psi(x,t),
\]
where:
\begin{itemize}
\item $\Psi(x,t)$ is the wave function,
\item $U(x)$ is the potential energy, and
\item $\frac{\partial^2}{\partial x^2}$ represents the second spatial derivative (kinetic energy operator).
\end{itemize}

\section{Separation of Variables in TDSE}
If the potential $U(x)$ is time-independent, the wave function can be expressed as:
\[
\Psi(x,t) = \sum_{n=1}^\infty c_n(t)\psi_n(x),
\]
where $\{\psi_n(x)\}$ forms an orthonormal basis of eigenfunctions of the time-independent Schrödinger equation:
\[
-\frac{\hbar^2}{2m} \frac{d^2\psi_n(x)}{dx^2} + U(x)\psi_n(x) = E_n\psi_n(x).
\]
The eigenvalues $E_n$ are the energies corresponding to the eigenfunctions $\psi_n(x)$.


\section{Time Dependence of Coefficients}
The time evolution of the coefficients $c_n(t)$ is determined by substituting $\Psi(x,t)$ into the TDSE:
\[
i \hbar \frac{\partial}{\partial t} \Psi(x,t) = \sum_{n=1}^\infty i \hbar \frac{dc_n(t)}{dt} \psi_n(x).
\]
On the right-hand side, we have:
\[
\hat{H}\Psi(x,t) = \sum_{n=1}^\infty c_n(t)E_n\psi_n(x),
\]
where $\hat{H} = -\frac{\hbar^2}{2m} \frac{\partial^2}{\partial x^2} + U(x)$.

Equating terms yields:
\[
i \hbar \frac{dc_n(t)}{dt} = E_n c_n(t),
\]
which simplifies to the ordinary differential equation:
\[
\frac{dc_n(t)}{dt} = -\frac{i E_n}{\hbar} c_n(t).
\]
The solution is:
\[
c_n(t) = c_n(0)e^{-iE_n t/\hbar}.
\]

\section{General Solution for $\Psi(x,t)$}
Substituting $c_n(t)$ into the wave function, we get:
\[
\Psi(x,t) = \sum_{n=1}^\infty c_n(0)e^{-iE_n t/\hbar}\psi_n(x).
\]

\section{Normalization of $\Psi(x,t)$}
The wave function $\Psi(x,t)$ must remain normalized:
\[
\int |\Psi(x,t)|^2 dx = 1.
\]
This is satisfied because the eigenfunctions $\{\psi_n(x)\}$ are orthonormal:
\[
\int \psi_n^*(x) \psi_m(x) dx = \delta_{nm},
\]
and the time-dependent coefficients $e^{-iE_n t/\hbar}$ preserve the norm.

\section{Example: Sinusoidal Initial State}
Let the initial state be:
\[
\Psi_0(x) = \frac{4}{\sqrt{5\pi}} \sin^3(x).
\]
We need to express $\Psi_0(x)$ in terms of the orthonormal basis $\psi_n(x) = \sqrt{\frac{2}{\pi}}\sin(nx)$. Using the trigonometric identity:
\[
\sin^3(x) = \frac{3}{4}\sin(x) - \frac{1}{4}\sin(3x),
\]
we find:
\[
\Psi_0(x) = \frac{3}{10}\psi_1(x) - \frac{1}{10}\psi_3(x),
\]
where the coefficients are:
\[
b_1 = \frac{3}{10}, \quad b_3 = -\frac{1}{10}, \quad b_n = 0 \text{ for } n \notin \{1,3\}.
\]

\section{Time Evolution of $\Psi(x,t)$}
The time-dependent solution is:
\[
\Psi(x,t) = \sum_{n=1}^\infty b_n e^{-iE_n t/\hbar} \psi_n(x).
\]
Substituting the coefficients and eigenfunctions:
\[
\Psi(x,t) = \frac{3}{10}e^{-iE_1t/\hbar}\psi_1(x) - \frac{1}{10}e^{-iE_3t/\hbar}\psi_3(x).
\]
Using $E_n = \frac{\hbar^2n^2}{2m}$, the final result becomes:
\[
\Psi(x,t) = 
\]
\[
\frac{3}{10} e^{-i\frac{\hbar t}{2m}} \sqrt{\frac{2}{\pi}} \sin(x) - \frac{1}{10} e^{-i\frac{9\hbar t}{2m}} \sqrt{\frac{2}{\pi}} \sin(3x).
\]

\section{Orthogonality Check}
To verify that $\psi_n(x)$ is an orthonormal basis:
\[
\int_0^\pi \psi_n(x)\psi_m(x) dx =
\begin{cases} 
1 & n = m, \\
0 & n \neq m.
\end{cases}
\]

\section{Key Formulae}
\begin{itemize}
\item Matrix exponential: For a diagonalizable matrix $\mathbf{A}$:
\[
e^{\mathbf{A}} = \mathbf{P} e^{\mathbf{D}} \mathbf{P}^{-1},
\]
where $\mathbf{D}$ is the diagonal matrix of eigenvalues and $\mathbf{P}$ is the eigenvector matrix.
\item Exponential series:
\[
e^x = \sum_{n=0}^\infty \frac{x^n}{n!}.
\]
\end{itemize}

\section{Introduction to the Time-Dependent Schrödinger Equation}
The time-dependent Schrödinger equation (TDSE) describes the evolution of the wave function $\Psi(x,t)$ in quantum mechanics:
\[
i\hbar \frac{\partial \Psi(x,t)}{\partial t} = \left( -\frac{\hbar^2}{2m} \frac{\partial^2}{\partial x^2} + U(x) \right) \Psi(x,t)
\]
Here:
\begin{itemize}
    \item $i = \sqrt{-1}$ is the imaginary unit.
    \item $\hbar$ is the reduced Planck constant.
    \item $U(x)$ is the potential energy.
    \item $\Psi(x,t)$ is the wave function, a complex-valued function encoding the probability distribution of the particle.
\end{itemize}

We consider an infinite potential well:
\[
U(x) = 
\begin{cases} 
0 & 0 \leq x \leq \pi, \\
\infty & \text{otherwise.}
\end{cases}
\]
The particle is confined to the region $0 \leq x \leq \pi$ and cannot exist elsewhere.

\section{Stationary States in the Infinite Well}
For a time-independent potential, we solve the time-independent Schrödinger equation:
\[
-\frac{\hbar^2}{2m} \frac{d^2 \psi_n(x)}{dx^2} = E_n \psi_n(x), \quad 0 \leq x \leq \pi.
\]
The boundary conditions $\psi_n(0) = \psi_n(\pi) = 0$ lead to the normalized eigenfunctions:
\[
\psi_n(x) = \sqrt{\frac{2}{\pi}} \sin(n x), \quad n = 1, 2, 3, \ldots.
\]
The corresponding energy eigenvalues are:
\[
E_n = \frac{\hbar^2 n^2}{2m}.
\]

\section{Initial State of the System}
The initial wave function is given as:
\[
\Psi_0(x) = \frac{4}{\sqrt{5\pi}} \sin^3(x).
\]
We need to express $\Psi_0(x)$ as a linear combination of the eigenfunctions $\{\psi_n(x)\}$:
\[
\Psi_0(x) = \sum_{n=1}^\infty b_n \psi_n(x).
\]

\section{Calculating the Fourier Coefficients $b_n$}
The coefficients $b_n$ are determined using the orthonormality property of $\psi_n(x)$:
\[
b_n = \int_0^\pi \Psi_0(x) \psi_n(x) dx.
\]
To simplify, we use the trigonometric identity for $\sin^3(x)$:
\[
\sin^3(x) = \frac{3}{4}\sin(x) - \frac{1}{4}\sin(3x).
\]
Substituting into $\Psi_0(x)$:
\[
\Psi_0(x) = \frac{4}{\sqrt{5\pi}} \left( \frac{3}{4}\sin(x) - \frac{1}{4}\sin(3x) \right).
\]
Rewriting in terms of $\psi_n(x)$:
\[
\Psi_0(x) = \frac{1}{\sqrt{10}} \left( 3\psi_1(x) - \psi_3(x) \right).
\]
This gives:
\[
b_1 = \frac{3}{\sqrt{10}}, \quad b_3 = -\frac{1}{\sqrt{10}},
\]
\[
\quad b_n = 0 \text{ for } n \notin \{1, 3\}.
\]

\section{Normalization of $\Psi_0(x)$}
To confirm that $\Psi_0(x)$ is normalized:
\[
\|\Psi_0\|^2 = \int_0^\pi |\Psi_0(x)|^2 dx = \sum_{n=1}^\infty |b_n|^2.
\]
Using $b_1 = \frac{3}{\sqrt{10}}$ and $b_3 = -\frac{1}{\sqrt{10}}$:
\[
\|\Psi_0\|^2 = \left( \frac{3}{\sqrt{10}} \right)^2 + \left( -\frac{1}{\sqrt{10}} \right)^2 = \frac{9}{10} + \frac{1}{10} = 1.
\]
Thus, $\Psi_0(x)$ is properly normalized.

\section{Time Evolution of the Wave Function}
The solution to the TDSE is:
\[
\Psi(x,t) = \sum_{n=1}^\infty b_n e^{-iE_n t/\hbar} \psi_n(x).
\]
Substituting the known $b_n$ and $E_n = \frac{\hbar^2 n^2}{2m}$:
\[
\Psi(x,t) = b_1 e^{-i \frac{\hbar t}{2m}} \psi_1(x) + b_3 e^{-i \frac{9\hbar t}{2m}} \psi_3(x).
\]
Simplifying:
\[
\Psi(x,t) = \frac{3}{\sqrt{10}} e^{-i \frac{\hbar t}{2m}} \sqrt{\frac{2}{\pi}} \sin(x) -
\]
\[
\frac{1}{\sqrt{10}} e^{-i \frac{9\hbar t}{2m}} \sqrt{\frac{2}{\pi}} \sin(3x).
\]

\section{Final Solution for $\Psi(x,t)$}
The complete wave function is:
\[
\Psi(x,t) = \frac{3}{\sqrt{5\pi}} \sin(x) e^{-i \frac{\hbar t}{2m}} - \frac{1}{\sqrt{5\pi}} \sin(3x) e^{-i \frac{9\hbar t}{2m}}.
\]
This describes the time evolution of the system in the infinite well.

\section{Key Observations}
\begin{itemize}
    \item The coefficients $b_n$ determine the contribution of each eigenstate $\psi_n(x)$.
    \item Only $\psi_1(x)$ and $\psi_3(x)$ contribute to $\Psi(x,t)$ because $\Psi_0(x)$ contains only $\sin(x)$ and $\sin(3x)$.
    \item The time dependence of each term is governed by the corresponding energy eigenvalue $E_n$.
\end{itemize}

\end{multicols}
\end{document}
